\chapter{总结与展望}

\section{全文总结}

随着云计算的兴起和通用处理器性能提升的放缓,基于 FPGA 的可编程网卡在数据中心被广泛部署。
本文提出了一个基于可编程网卡的高性能数据中心系统,即利用可编程网卡提高云计算中计算、网络和存储节点的性能,并降低成本。

首先,我们提出用基于FPGA的可编程网卡加速云计算中的虚拟网络功能。为了简化编程,提出了首个适用于高速网络数据包处理、基于高级语言的FPGA编程框架。相比传统基于CPU的网络功能,把吞吐量提高了10倍,延迟降低到1/10,还为每个计算节点节约了1/5的CPU核。

其次,我们提出用可编程网卡加速远程数据结构访问。我们以内存键值存储系统为例,在服务器端绕过CPU,用网卡直接访问主机内存,实现了10倍于CPU的吞吐量和微秒级的延迟,是首个单机性能达到10亿次每秒的通用键值存储系统。

最后,为了降低软件网络协议栈的开销,我们提出一个软硬件结合的套接字网络协议栈,与现有应用程序完全兼容,并能实现接近硬件极限的吞吐量和延迟,解决了长期以来通用协议栈性能较低、专用协议栈兼容性较差的矛盾。


\section{未来工作展望}

\iffalse
探索的动机
性能优化:节约成本,满足客户需求。
首先,消极的动机:简化系统架构、简化编程,减少设计中的权衡。延迟 budget,可以做更复杂的计算,可以更精确。可以不加入额外的 cache,可以不重造轮子(如修改应用使用 RDMA)。
还有一种积极的动机:赋能更多应用。
系统研究者的最高使命:提出普遍抽象。
渴望达到硬件极限,是无穷的毅力和耐心的源泉。
\fi