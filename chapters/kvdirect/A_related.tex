\section{相关工作}
\label{kvdirect:sec:related}

作为一种重要的基础设施,分布式键值存储系统的研究和开发受到性能的驱动。大量分布式键值存储基于CPU。为了降低计算成本,Masstree~ \cite {mao2012cache},MemC3~ \cite {fan2013memc3}和libcuckoo~ \cite {li2014algorithmic}优化锁定,缓存,散列和内存分配算法,而 \oursys{} 带有新的哈希值专为FPGA设计的表和内存管理机制,以最大限度地减少PCIe流量。 MICA~ \cite {lim2014mica}将哈希表分区到每个核心,从而完全避免同步。然而,这种方法为偏移的工作负载引入了核心不平衡。

为了摆脱操作系统内核的开销, \cite {rizzo2012netmap,intel2014data}直接轮询来自网卡的网络数据包,并且\cite {jeong2014mtcp,marinos2014network}使用用户空间轻量级网络堆栈处理它们。
键值存储系统 \cite {kapoor2012chronos,ousterhout2010case,ousterhout2015ramcloud,lim2014mica,li2016full}受益于这种高性能优化。
作为朝着这个方向迈出的又一步,最近的作品\cite {infiniband2000infiniband,kalia2014using,kalia2016design,kalia2014using,kalia2016design}利用RDMA 网卡的基于硬件的网络堆栈,使用双面RDMA作为键值存储客户端和服务器之间的RPC机制进一步提高每核吞吐量并减少延迟。尽管如此,这些系统仍受CPU限制(\S \ref {kvdirect:sec:state-of-the-art-kvs})。

另一种不同的方法是利用单侧RDMA。 Pilaf~ \cite {mitchell2013using}和FaRM~ \cite {dragojevic2014farm}采用单向RDMA读取进行GET操作,FaRM实现了使网络饱和的吞吐量。 Nessie~ \cite {szepesi2014designing},DrTM~ \cite {wei2015fast},DrTM + R~ \cite {chen2016fast}和FaSST~ \cite {kalia2016fasst}利用分布式事务来实现单向RDMA的GET和PUT。但是,PUT操作的性能受到一致性保证的不可避免的同步开销的影响,受到RDMA原语的限制 \cite {kalia2016design}。此外,客户端CPU涉及键值处理,将每个核心的吞吐量限制在客户端的大约10~Mops。
相比之下,\oursys{} 将RDMA原语扩展到键值操作,同时保证服务器端的一致性,使键值存储客户端完全透明,同时实现高吞吐量和低延迟,即使对于PUT操作也是如此。

作为一种灵活且可定制的硬件,FPGA现已广泛部署在数据中心规模 \cite {putnam2014reconfigurable,caulfield2016cloud}中,并且针对可编程性进行了大幅改进 \cite {bacon2013fpga,li2016clicknp}。一些早期的工作已探索在FPGA上构建键值存储。但是其中一些是不实用的,限制片上数据存储(大约几MB内存) \cite {liang16fpl}或板载DRAM(通常是8GB内存) \cite {istvan2013flexible,chalamalasetti2013fpga,istvan2015hash}。
 \cite {blott2015scaling}专注于提高系统容量而不是吞吐量,并采用SSD作为板载DRAM的二级存储。
 \cite {liang16fpl,chalamalasetti2013fpga}限制它们在固定大小键值对中的使用,这只能用于特殊用途,而不是一般的键值存储。
\cite {blott13hotcloud,lavasani2014fpga}使用主机DRAM存储哈希表,而\cite {tokusashi2016multilevel}使用网卡 DRAM作为主机DRAM的缓存,但它们没有针对网络和PCIe DMA带宽进行优化,导致性能不佳。
\oursys{} 充分利用了网卡 DRAM和主机DRAM,使我们基于FPGA的键值存储系统通用,并且能够进行大规模部署。此外,我们精心的硬件和软件协同设计,以及对PCIe和网络的优化,将性能推向了物理限制,推进了最先进的解决方案。

二级索引是通过其他键检索数据的重要功能
比数据存储系统中的主键 \cite {escriva2012hyperdex,kejriwal2016slik}。 SLIK~ \cite {kejriwal2016slik}在键值存储系统中使用B +树算法支持多个二级密钥。探索如何支持二级索引以帮助 \oursys{} 迈向通用数据存储系统将会很有趣。 Switch键值~ \cite {li2016fast}利用基于内容的路由将请求路由到基于缓存密钥的后端节点,NetCache~ \cite {netcache-sosp17}进一步将键值缓存在交换机中。这种负载平衡和缓存也将使我们的系统受益。
Eris~ \cite {eris}利用网络序列发生器来实现高效的分布式事务,这可以为客户端同步的单侧RDMA方法带来新的生命。


%\textbf{TODO}

%SSD or HDD for persistence~\cite{marmol2014nvmkv, debnath2010flashstore, papagiannis2016tucana}
%network switch~\cite{li2016fast}


%main limitations:

%~\cite{liang16fpl} only uses on-chip BRAM.
%~\cite{istvan2013flexible,chalamalasetti2013fpga,istvan2015hash} only use on-board DRAM, These work don't make use of host RAM, result in small system capacity.

%~\cite{lavasani2014fpga, blott13hotcloud} leverages host RAM, however they didn't have optimization for PCIe DMA which becomes the bottleneck.

%~\cite{liang16fpl, lavasani2014fpga} only support fixed-size Key-value pair, not pratical.


%all works lack of optimization for network like batching and only achieves low encoding efficiency, resulting in low network bandwidth utilization.
%all works lack of optimization for hashtable, which leads to higher average search in heavy load.
