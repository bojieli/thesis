\section{设计}
\label{kvdirect:sec:architecture}

\subsection{系统架构}

KV-Direct使\textit {远程直接键值访问}成为可能。
客户端将\textit {KV-Direct 操作}(\S \ref {kvdirect:sec:kv-operations})发送到KVS服务器,而可编程网卡处理请求并发送回结果,绕过CPU。
KVS服务器上的可编程网卡是一个重新配置为\textit {KV处理器}的FPGA(\S \ref {kvdirect:sec:kv-processor})。
图\ref {kvdirect:fig:kvdirect-arch}显示了KV-Direct的架构。

\subsection{KV-Direct Operations}
\label{kvdirect:sec:kv-operations}

\begin{table}
\centering
\caption{KV-Direct operations.}
\label{kvdirect:tab:kv-operations}

\small
\begin{tabular}{p{.4\textwidth}|p{.5\textwidth} }
\toprule
get ($k$) $\rightarrow v$ & Get the value of key $k$. \\
\midrule
put ($k, v$) $\rightarrow$ bool & Insert or replace a $(k, v)$ pair. \\
\midrule
delete ($k$) $\rightarrow$ bool & Delete key $k$. \\
\midrule
\midrule
update{\_}scalar2scalar ($k, \Delta, \lambda(v, \Delta) \rightarrow v$) $\rightarrow v$ & Atomically update the value of key~$k$ using function~$\lambda$ on scalar~$\Delta$, and return the original value. \\
\midrule
update{\_}scalar2vector ($k, \Delta, \lambda(v, \Delta) \rightarrow v$) $\rightarrow [v]$ & Atomically update all elements in vector~$k$ using function~$\lambda$ and scalar~$\Delta$, and return the original vector. \\
\midrule
update{\_}vector2vector ($k, [\Delta], \lambda(v, \Delta) \rightarrow v$) $\rightarrow [v]$ & Atomically update each element in vector~$k$ using function~$\lambda$ on the corresponding element in vector~$[\Delta]$, and return the original vector. \\
\midrule
reduce ($k, \Sigma, \lambda(v, \Sigma) \rightarrow \Sigma$) $\rightarrow \Sigma$ & Reduce vector~$k$ to a scalar using function~$\lambda$ on initial value, and return the reduction result~$\Sigma$. \\
\midrule
filter ($k, \lambda(v) \rightarrow$ bool) $\rightarrow [v]$ & Filter elements in a vector~$k$ by function~$\lambda$, and return the filtered vector. \\
\bottomrule
\end{tabular}

\end{table}

KV-Direct将单面RDMA操作扩展到键值操作,如表\ref {kvdirect:tab:kv-operations}中所述。
除了表\ref {kvdirect:tab:kv-operations}顶部所示的标准KVS操作外,KV-Direct还支持两种类型的向量运算:
将标量发送到服务器上的NIC,NIC将更新应用于向量中的每个元素;或者向服务器发送一个向量,并且NIC逐个元素地更新原始向量。
此外,KV-Direct支持用户定义的更新功能,作为原子操作的概括。
更新功能需要在执行前预先注册并编译为硬件逻辑。
使用用户定义的更新功能的KV操作类似于\textit {动态消息}(active messages)\cite {eicken1992active},从而节省了通信和同步成本。

当对键执行向量操作更新(update),归约(reduce)或过滤(filter)时,其值被视为固定位宽元素的数组。
每个函数 $\lambda$ 对向量中的一个元素,客户端指定的参数 $\Delta$ 和/或初始值 $\Sigma$ 进行归约操作。
KV-Direct开发工具链多次复制$\lambda$以利用FPGA中的并行性并将计算吞吐量与PCIe吞吐量相匹配,然后使用高级综合(HLS)工具将其编译为可重新配置的硬件逻辑\cite {aoc} 。
HLS工具自动提取复制函数中的数据依赖性,并生成完全流水线的可编程逻辑。
在执行KVS客户端之前,KVS服务器上的可编程NIC应加载包含展开的$\lambda$的硬件逻辑。

使用用户定义函数的更新操作能够对矢量值进行常规流处理。
例如,网络处理应用程序可以将该向量解释为用于网络功能的分组流\cite{li2016clicknp}或用于分组事务的一堆状态\cite {sivaraman2016packet}。
完全在可编程NIC中的单对象事务处理也是可能的,例如,在TPC-C基准中包围S\_QUANTITY \cite {council2010tpc}。
向量归约操作支持PageRank \cite {page1999pagerank}中的邻居权重累积。
可以使用向量过滤操作来获取稀疏向量中的非零值。

\subsection{KV 处理器}
\label{kvdirect:sec:kv-processor}

\begin{figure}[t]
\centering
\includegraphics[width=0.6\textwidth,page=1]{processor_architecture.PNG}
\caption{KV 处理器架构。}
\label{kvdirect:fig:kvprocessor-arch}

\end{figure}

如图\ref {kvdirect:fig:kvprocessor-arch}所示,KV处理器从板载NIC接收数据包,解码向量操作并缓冲保留站中的KV操作(\S \ref {kvdirect:sec:ooo})。
接下来,乱序执行引擎(\S \ref {kvdirect:sec:ooo})从保留站向操作解码器发出独立的KV操作。
根据操作类型,KV处理器查找哈希表(\S \ref {kvdirect:sec:hashtable})并执行相应的操作。
为了最小化内存访问次数,较小的KV对在哈希表中内联(inline)存储,其他的KV对存储在slab内存分配器(\S \ref {kvdirect:sec:slab})的动态分配内存中。
散列索引和slab分配的内存都由统一的内存访问引擎(\S \ref {kvdirect:sec:dram-cache})管理,它通过PCIe DMA访问主机内存并缓存部分主机内存。板载DRAM。
在KV操作完成之后,结果被发送回无序执行引擎(\S \ref {kvdirect:sec:ooo})以在保留站中找到并执行匹配的KV操作。

正如\S \ref {kvdirect:sec:challenge}中所讨论的,PCIe操作吞吐量的稀缺性要求KV处理器在DMA访问上节俭。
对于GET操作,至少需要读取一次内存。
对于PUT或DELETE操作,对于哈希表,一次读取和一次写入最小。
基于日志的数据结构可以实现每个PUT一次写入,但它牺牲了GET性能。
KV-Direct仔细设计哈希表,以便在每次查找和插入时实现接近理想的DMA访问,以及内存分配器。每次动态内存分配平摊下来,只需不到0.1次DMA操作。

\subsubsection{哈希表}
\label{kvdirect:sec:hashtable}

\begin{figure}[t]
\centering
\includegraphics[width=1.0\textwidth,page=1]{hashline.PNG}
\caption{哈希索引结构。 每行是一个哈希桶,包含10个哈希槽,每个哈希槽3位板内存类型,一个位图标记内联KV对的开始和结束,以及指向哈希冲突下一个链接桶的指针。}
\label{kvdirect:fig:hashtable}

\end{figure}

为了存储可变大小的KV,KV存储分为两部分。 第一部分是哈希索引(图\ref {kvdirect:fig:hashtable}),它包含固定数量的\textit {哈希桶}。 每个哈希桶包含几个\textit {哈希槽}和一些元数据。 内存的其余部分是动态分配的,由slab分配器(\S \ref {kvdirect:sec:slab})管理。
初始化时配置的\textit {哈希索引比率}确定为哈希索引分配的内存百分比。
哈希索引比率的选择将在\S \ref {kvdirect:sec:hashtable-eval}中讨论。

%\textbf{Hash Table.}
%Each bucket includes 10 hash slots, 3b type code per slot, 50b metadata, plus 31b address and a valid bit of the next chained bucket, as shown in Figure~\ref{kvdirect:fig:hashtable}.
%For offline KVs, each hash slot needs to store 31b of address, 9b of secondary hash and 3b type code for the slab size.
%For inline KVs, to mark the begin and end of each hash slot, as well as the separation between inline key and value, the information is encoded in a 50b metadata corresponding to 50 bytes of hash slots.
%The inline keys and secondary hashes of offline keys in all hash slots are compared in parallel, and the first match is found.

每个散列槽包括指向动态分配的存储器中的KV数据的指针和辅助散列。
辅助哈希是一种启用并行内联检查的优化。始终检查密钥以确保正确性,但需要额外的一次内存访问。
假设主机内存中的64~GiB KV存储和32字节分配粒度(内部碎片和分配元数据开销之间的权衡),指针需要31位。
9位的二级散列给出1/512误报可能性。
累积地,散列槽大小是5个字节。
为了确定散列桶大小,我们需要在每个桶的散列槽数和DMA吞吐量之间进行权衡。
图\ref {kvdirect:fig:dma-tput}表明DMA读取吞吐量低于64B粒度受DMA引擎中PCIe延迟和并行性的约束。
由于哈希冲突的可能性增加,小于64B的桶大小是次优的。
另一方面,将桶大小增加到64B以上会降低散列查找吞吐量。
因此我们选择桶大小为64字节。

\begin{figure}[t]
\centering
\includegraphics[width=0.6\textwidth]{inline_thresh.pdf}
\caption{Average memory access count under varying inline thresholds (10B, 15B, 20B, 25B) and memory utilizations.}
\label{kvdirect:fig:inline-offline}

\end{figure}

\textit {KV 大小}是指键和值的总大小。
小于阈值的KV在哈希索引中内联(inline)存储,以保存对获取KV数据的额外存储器访问。
内联KV可以跨越多个散列槽,其指针和二级散列字段被重新用于存储KV数据。
内联所有可装入铲斗的KV可能不是最佳选择。
为了最小化平均访问时间,假设可以同等地访问越来越小的密钥,则更希望内联小于\textit {内联阈值}的KV。
为了量化所有桶中使用的桶的部分,我们使用\textit {内存利用率}而不是负载率(load factor),因为它更多地涉及可以适合固定内存量的KV的数量。
如图\ref {kvdirect:fig:inline-offline}所示,对于某个内联阈值,由于更多的哈希冲突,平均内存访问计数随内存利用率的增加而增加。
较高的内联阈值显示内存访问计数的更陡峭的增长曲线,因此可以找到最佳内联阈值以最小化在给定内存利用率下的内存访问。
与哈希索引比率一样,也可以在初始化时配置内联阈值。

当存储桶中的所有插槽都已填满时,有几种解决方案可以解决散列冲突。
Cuckoo哈希\cite {pagh2004cuckoo}和跳房子哈希(Hopscotch Hash)\cite {herlihy2008hopscotch}通过在插入过程中移动占用的插槽来保证恒定时间查找。
但是,在写入密集型工作负载中,高负载率下的内存访问时间会经历大的波动。
线性探测可能受到主群集的影响,因此其性能对散列函数的均匀性敏感。
我们选择\textit {拉链法}来解决哈希冲突,这会平衡查找和插入,同时对哈希群集更加健壮。

%In KV-Direct, we measure memory utilization instead of load factor, because chaining has dynamic size and that we care more about the overall storage efficiency counting all indexing, metadata and memory fragmentation overhead.
%Clearly, small KVs cause lower memory utilization due to metadata overhead.
%The optimal hash index ratio is chosen at initialization time according to workload to balance average access time and memory utilization.

\subsubsection{Slab Memory Allocator}
\label{kvdirect:sec:slab}

Chained hash slots and non-inline KVs need dynamic memory allocation.
We choose slab memory allocator~\cite{bonwick1994slab} to achieve $O(1)$ average memory access per allocation and deallocation. The main slab allocator logic runs on host CPU and communicates with the KV-processor through PCIe.
Slab allocator rounds up allocation size to the nearest power of two, called \textit{slab size}.
It maintains a \textit{free slab pool} for each possible slab size (32, 64, \ldots, 512 bytes), and a global \textit{allocation bitmap} to help to merge small free slabs back to larger slabs.
Each free slab pool is an array of \textit{slab entries} consisting of an address field and a slab type field indicating the size of the slab entry.
The free slab pool can be cached on the NIC. The cache syncs with the host memory in batches of slab entries. Amortized by batching, less than 0.07 DMA operation is needed per allocation or deallocation.
When a small slab pool is almost empty, larger slabs need to be split.
Because the slab type is already included in a slab entry, in \textit{slab splitting}, slab entries are simply copied from the larger pool to the smaller pool, without the need for computation.
Including slab type in the slab entry also saves communication cost because one slab entry may contain multiple slots.

On deallocation, the slab allocator needs to check whether the freed slab can be merged with its neighbor, requiring at least one read and write to the allocation bitmap.
Inspired by garbage collection, we propose \textit{lazy slab merging} to merge free slabs in batch when a slab pool is almost empty and no larger slab pools have enough slabs to split.

\subsubsection{Out-of-Order Execution Engine}
\label{kvdirect:sec:ooo}

\begin{figure}[t]
\centering
\includegraphics[width=.9\textwidth,page=1]{dynamic_scheduler.PNG}
\caption{Dynamic operation scheduler.}
\label{kvdirect:fig:ooo-mem-access}
\end{figure}

Dependency between two KV operations with the same key in the KV processor will lead to data hazard and pipeline stall.
This problem is magnified in single-key atomics where all operations are dependent, thus limiting the atomics throughput.
We borrow the concept of dynamic scheduling from computer architecture and implement a \textit{reservation station} to track all in-flight KV operations and their \textit{execution context}.
To saturate PCIe, DRAM and the processing pipeline, up to 256 in-flight KV operations are needed.
However, comparing 256 16-byte keys in parallel would take 40\% logic resource of our FPGA.
Instead, we store the KV operations in a small hash table in on-chip BRAM, indexed by the hash of the key.
To simplify hash collision resolution, we regard KV operations with the same hash as dependent, so there may be false positives, but it will never miss a dependency.
Operations with the same hash are organized in a chain and examined sequentially.
Hash collision would degrade the efficiency of chain examination, so the reservation station contains 1024 hash slots to make hash collision possibility below 25\%.

The reservation station not only holds pending operations, but also caches their latest values for \textit{data forwarding}.
When a KV operation is completed by the main processing pipeline, its result is returned to the client, and the latest value is forwarded to the reservation station.
Pending operations in the same hash slot are checked one by one, and operations with matching key are executed immediately and removed from the reservation station.
For atomic operations, the computation is performed in a dedicated execution engine.
For write operations, the cached value is updated.
The execution result is returned to the client directly.
After scanning through the chain of dependent operations, if the cached value is updated, a PUT operation is issued to the main processing pipeline for cache write back.
This data forwarding and fast execution path enable single-key atomics to be processed one operation per clock cycle (180~Mops), eliminate head-of-line blocking under workload with popular keys, and ensure consistency because no two operations on the same key can be in the main processing pipeline simultaneously.


\subsubsection{DRAM Load Dispatcher}
\label{kvdirect:sec:dram-cache}

\begin{figure}[t]
\centering
\includegraphics[width=0.8\textwidth,page=1]{load_balancer.PNG}
\caption{DRAM load dispatcher.}
\label{kvdirect:fig:cache}

\end{figure}

To further save the burden on PCIe, we dispatch memory accesses between PCIe and the NIC on-board DRAM.
Our NIC DRAM has 4~GiB size and 12.8~GB/s throughput, which is an order of magnitude smaller than the KVS storage on host DRAM (64~GiB) and slightly slower than the PCIe link (14~GB/s).
One approach is to put a fixed portion of the KVS in NIC DRAM. However, the NIC DRAM is too small to carry a significant portion of memory accesses.
The other approach is to use the NIC DRAM as a cache for host memory, the throughput would degrade due to the limited throughput of our NIC DRAM.

We adopt a hybrid solution to use the DRAM as a cache for a fixed portion of the KVS in host memory, as shown in Figure~\ref{kvdirect:fig:cache}.
The cache-able portion is determined by the hash of memory address, in granularity of 64 bytes. The hash function is selected so that an address in hash index and dynamically allocated memory have an equal possibility of being cache-able.
The portion of cache-able memory in the entire KVS memory is called \textit{load dispatch ratio} ($l$).
With a larger load dispatch ratio $l$, more load is dispatched to the on-board DRAM and the cache hit ratio $h(l)$ would increase.
Assume the cache hit possibility is $h(l)$.
To balance load on PCIe and on-board DRAM, the load dispatch ratio $l$ should be optimized so that:
$$\frac{l}{tput_{DRAM}} = \frac{(1-l) + l \cdot (1-h(l))}{tput_{PCIe}}$$

Specifically, under uniform workload, let $k$ be the ratio of on-board DRAM size and host KV storage size, cache hit possibility $h(l) = \frac{\textnormal{cache size}}{\textnormal{cache-able memory size}} = \frac{k}{l}$ when $k \leq l$.
Caching under uniform workload is not efficient.
Under long-tail workload with Zipf distribution, assume $n$ is the total number of KVs, approximately $h(l) = \frac{\log (\textnormal{cache size})}{\log (\textnormal{cache-able corpus size})} = \frac{\log (kn)}{\log (ln)}$ when $k \leq l$.
Under long-tail workload, the cache hit possibility as high as 0.7 with 1M cache in 1G corpus.
An optimal $l$ can be solved numerically, as shown in \S\ref{kvdirect:sec:different-nic}.

\egg{
\subsubsection{Congestion Avoidance}
\label{kvdirect:sec:congestion-avoidance}

In addition to throughput, another important factor is latency.
From the client's perspective, the KV processor is a path with multiple bottlenecks and buffers, \eg, PCIe and DRAM access.
If all buffers in the KV processor are filled up, the GET latency would exceed 10~$\mu$s.
To mitigate the bufferbloat problem, we implement a congestion avoidance logic to limit the number of in-flight KV operations \textit{inside the KV processor}.
The KV processor maintains a \textit{KV operation window} and leverages credit-based flow control mechanism in RDMA to back-pressure KVS clients.
To adapt KV operation window size to the workload, we measure the running average of KV processing delay and adjust the window size according to TCP Vegas congestion avoidance algorithm~\cite{brakmo1995tcp}.
%We use delay as the congestion signal instead of ECN, because the queues whose sizes are hard to measure.
}
