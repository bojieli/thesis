\section{本章小结}
\label{kvdirect:sec:conclusion}


本章描述了\oursys {}的设计和评估,这是一个高性能的内存中键值存储。 在计算机系统设计的悠久历史中,\oursys {}是另一项利用可重新配置硬件来加速重要工作量的练习。 \oursys {}能够通过精心设计硬件和软件来获得卓越的性能,以消除系统中的瓶颈并实现接近底层硬件物理极限的性能。

\iffalse
\textbf{从网络到数据结构}
\fi



\begin{sidewaystable}[h]
	\centering
	\caption{KV-Direct 与其他键值存储系统在长尾(倾斜  Zipf)负载和 10 字节小键下的比较。对于相关工作未报告的性能数字,本文使用相似的硬件来模拟这些系统,并报告粗略的测量结果。对于 CPU 绕过的系统,括号内的数字报告峰值负载和空闲情况下的功耗差异。}
	\label{kvdirect:tab:kvs-compare}
	\small
	\begin{tabular}{|l|l|l|r|r|r|r|r|r|r|}
		\toprule
		键值存储  & 注释 & 性能瓶颈 & \multicolumn{2}{c|}{吞吐量 (Mops)} & \multicolumn{2}{c|}{功耗效率 (Kops/W)} & \multicolumn{2}{c|}{平均延迟 ($\mu$s)} \\
		\cline{4-9}表 \ref {clicknp:tab:elements} 
		& & & GET & PUT & GET & PUT & GET & PUT \\
		\midrule
		Memcached~\cite{fitzpatrick2004distributed} & 传统 & CPU 核间同步 & 1.5 & 1.5 & \approx5 & \approx5 & \approx50 & \approx50 \\
		MemC3~\cite{fan2013memc3} & 传统 & 操作系统网络协议栈 & 4.3 & 4.3 & \approx14 & \approx14 & \approx50 & \approx50 \\
		RAMCloud~\cite{ousterhout2015ramcloud} & 内核绕过 & 分派线程 & 6 & 1 & \approx20 & \approx3.3 & 5 & 14 \\
		MICA~\cite{lim2014mica} & 内核绕过,24 个核,12 个网卡口 & CPU 键值处理 & 137 & 135 & 342 & 337 & 81 & 81 \\
		FaRM~\cite{dragojevic2014farm} & 单边 RDMA GET & RDMA 网卡 & 6 & 3 & \approx30 (261) & \approx15 & 4.5 & \approx10 \\
		DrTM-KV~\cite{wei2015fast} & 单边 RDMA 和 HTM & RDMA 网卡 & 115.2 & 14.3 & \approx500 (3972) & \approx60 & 3.4 & 6.3 \\
		HERD'16~\cite{kalia2016design} & 双边 RDMA, 12 核 & PCIe & 98.3 & \approx60 & \approx490 & \approx300 & 5 & 5 \\
		Xilinx'13~\cite{blott13hotcloud} & FPGA & 网络 & 13 & 13 & 106 & 106 & 3.5 & 4.5 \\
		Mega-KV~\cite{zhang2015mega} & GPU (4~GiB 板上 RAM) & GPU 键值处理 & 166 & 80 & \approx330 & \approx160 & 280 & 280 \\
		\midrule
		\textbf{KV-Direct (1 网卡)} & 可编程网卡,两个 Gen3 x8 & PCIe \& DRAM & 180 & 114 & 1487 (5454) & 942 (3454) & 4.3 & 5.4 \\
		\textbf{KV-Direct (10 网卡)} & 可编程网卡,每卡一个 Gen3 x8 & PCIe \& DRAM & 1220 & 610 & 3417 (4518) & 1708 (2259) & 4.3 & 5.4 \\
		\bottomrule
	\end{tabular}
\end{sidewaystable}
