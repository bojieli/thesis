%!TEX root=main.tex
\section{引言}
\label{kvdirect:sec:introduction}

本章的主题是存储虚拟化与数据结构处理加速,在全文中的位置如图 \ref{kvdirect:fig:sys-arch} 所示。

\begin{figure}[htbp]
	\centering
	\includegraphics[width=0.8\textwidth]{figure/sys_arch.pdf}
	\caption{本章主题:存储虚拟化与数据结构处理加速,用红色标出。}
	\label{kvdirect:fig:sys-arch}
\end{figure}

在可编程网卡的编程方面,本章基于上一章提出的 ClickNP 编程框架,搭建了服务层的基础,提出了有状态处理和数据结构处理的框架,并基于此实现了内存键值存储,如图 \ref{kvdirect:fig:sw-hw-codesign} 所示。

\begin{figure}[htbp]
	\centering
	\includegraphics[width=0.5\textwidth]{figure/sw_hw_codesign.pdf}
	\caption{本章在可编程网卡软硬件架构中的位置。}
	\label{kvdirect:fig:sw-hw-codesign}
\end{figure}


%\subsection{outline}
%
%键值 is an important infrastructure service
%
%Existing solution of key-value store in
%
%1. CPU-based
%
%2. RDMA-based
%
%recent trend of programmable 网卡 in datacenter
%
%we present KV-Direct... 
%
%highlight the high level insight (offload the simple and performance critical operation to programmable 网卡) 
%
%and design goal and spotlight (not only match the state of art of RDMA-based solution in GET operation and significant improve the PUT operation to even comparable to CPU-based solution. Moreover, our solution is more power and cost effective. With more FPGA or the improvement of FPGA, we can beat all existing CPU/RDMA solution.)
%
%KV-Direct makes contributions in three main aspects:
%
%1. proposes a 键值 solution for programmable 网卡 to Direct Memory Access.
%
%2. extends RDMA primitives to Key-value operation with consistency guarantee while high performance. 
%
%3. makes critical PCIe traffic optimization in our novel FPGA-based 网卡 key-value service design.



%Key-value store (键值存储) is an infrastructure service to store shared structured data in a distributed system.
%Traditionally, the performance of 键值存储 is mostly constrained by the OS network stack.
%Recent research leverage two-sided RDMA to accelerate communication among clients and 键值存储 servers.
%The next bottleneck is CPU. The 键值 operation throughput of a CPU core is constrained by its latency hiding efficiency and computation capacity.
%Another approach is to use one-sided RDMA to bypass remote CPU and move 键值 processing to clients, but it incurs high communication and synchronization overhead.
%\subsection{current version}




\iffalse
内存中键值存储是许多数据中心中的关键分布式系统组件。历史上,诸如 Memcached \cite {fitzpatrick2004distributed}的键值存储作为 Web 服务的对象缓存系统获得了普及。亚马逊 \cite {decandia2007dynamo} 和 Facebook \cite {atikoglu2012workload,nishtala2013scaling} 等大型网络服务提供商已大规模部署了分布式键值存储。最近,随着基于内存的计算成为数据中心的一个主要趋势 \cite{ousterhout2010case,dragojevic2014farm},键值存储开始超越缓存并成为在分布式系统中存储共享数据结构的基础架构。

许多数据结构可以在键值哈希表中表示,例如,NoSQL数据库中的数据索引 \cite {chang2008bigtable},机器学习中的模型参数 \cite {li2014scaling},图计算中的点和边 \cite {shao2013trinity,xiao17tux2} 和分布式同步中的序列发生 器\cite {kalia2016design}。对于大多数这些应用,键值存储的性能是直接决定系统效率的关键因素。由于其重要性,多年来已投入大量研究工作来改善键值存储性能。

早期的键值系统\cite {decandia2007dynamo,fitzpatrick2004distributed,nishtala2013scaling} 建立在传统操作系统抽象的基础之上,例如OS锁和TCP / IP堆栈。这给操作系统的性能带来了相当大的压力,尤其是网络堆栈。由于数据中心应用带宽需求过大,物理网络传输速度在过去十年中有了巨大的改进,这加剧了瓶颈。

最近,随着单核频率提升和多核架构扩展速度的减慢,分布式系统的一项新研究趋势是利用网卡上的远程直接内存访问(RDMA)技术来减少网络处理成本。一系列研究\cite {kalia2014using,kalia2016design}使用双面RDMA加速通信(图\ref {kvdirect:fig:memaccess-a})。使用这种方法构建的键值存储受到键值存储服务器的CPU性能的限制。另一系列研究使用单侧RDMA绕过远程CPU并将键值处理工作量转移到客户端\cite {dragojevic2014farm,mitchell2013using}(图\ref {kvdirect:fig:memaccess-b})。这种方法实现了更好的GET性能,但由于高通信和同步开销,降低了PUT操作的性能。由于缺乏事务支持,RDMA提供的抽象不适合构建高效的键值存储。

与此同时,数据中心硬件发展的另一个趋势正在出现。数据中心中越来越多的服务器现在配备了可编程网卡 \cite{caulfield2016cloud,greenberg2015sdn,putnam2014reconfigurable}。
可编程网卡的核心是现场可编程门阵列(FPGA),其具有用于连接到网络的嵌入式网卡芯片和用于连接到服务器的PCIe连接器。
可编程网卡最初设计用于启用网络虚拟化\cite {vfp,li2016clicknp}。
但是,许多人发现FPGA资源可以用来卸载CPU的一些工作负载,并显着降低CPU资源的使用量\cite {ouyang14hotchips,MaZC17fpga,huang16socc,cong16dac}。本章采用这种一般方法。
\fi

内存键值存储是数据中心的关键分布式系统组件。
本章提出 \oursys{},一个基于可编程网卡的内存键值系统。
顾名思义,\oursys{} 的可编程网卡从网络上接收并处理键值操作请求,并直接在主机内存中应用更新,绕过主机CPU。
\oursys{} 将RDMA原语从内存操作(读和写)扩展到键值操作(GET,PUT,DELETE和原子操作)。
此外,为了支持基于向量的操作并减少网络流量,\oursys{} 还提供了新的向量原语UPDATE,REDUCE和FILTER,允许用户定义活动消息(active message)\cite {eicken1992active} 并将某些计算委托给可编程网卡。

在可编程网卡内进行键值处理的设计重点是优化网卡和主机内存之间的PCIe流量。
\oursys{} 采用一系列优化来充分利用PCIe带宽和隐藏延迟。
首先,\oursys{} 设计了一个新的哈希表和内存分配器,以利用FPGA的并行性,并最大限度地减少PCIe DMA请求的数量。
平均而言,\oursys{} 每次 GET 操作仅使用接近一次 PCIe DMA 操作,每次 PUT 操作仅使用两次 PCIe DMA 操作。
其次,为了保证键值存储的一致性,\oursys{} 设计了一个乱序执行引擎来跟踪操作依赖性,同时最大化独立请求的吞吐量。
第三,\oursys{} 在 FPGA 中实现了基于硬件的负载分派程序和缓存组件,以充分利用板载 DRAM 带宽和容量。

基于上述优化,单网卡 \oursys{} 系统能够实现每秒高达 180~M 次键值操作,相当于 36 个 CPU 核心的吞吐量 \cite {li2016full}。
与最先进的 CPU 键值存储系统相比,\oursys{} 可将尾延迟降低至 10 $\mu$s,同时将能耗效率提高3倍。
而且,\oursys{} 可以通过多个网卡实现接近线性的可扩展性。通过在单台商品服务器中使用 10 个可编程网卡,性能可达每秒 12.2 亿键值操作,这比现有系统提高了一个数量级。

\oursys{} 还支持高达 180~Mops 的通用原子操作,明显优于目前最先进的基于 RDMA 的系统中报告的性能:2.24~Mops \cite {kalia2014using}。原子操作的高性能主要归功于乱序执行引擎。乱序执行引擎可以高效地跟踪键值操作之间的依赖性,而不会阻塞流水线。

本章的其余部分安排如下。第 \ref {kvdirect:sec:background} 节介绍背景,并阐明设计目标和挑战。
第 \ref {kvdirect:sec:architecture} 节描述 \oursys{} 的设计。
第 \ref {kvdirect:sec:evaluation} 节评估 \oursys{} 的性能。
第 \ref {kvdirect:sec:extensions} 节讨论进一步的扩展。
第 \ref {kvdirect:sec:related} 节讨论相关工作。
第 \ref {kvdirect:sec:conclusion} 节总结。
