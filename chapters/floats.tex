% !TeX root = ../main.tex

\chapter{Floating Objects}

\section{Three-line Table}

The three-line table is the format recommended by the "Writing Manual", as shown in Table~\ref{tab:exampletable}.
\begin{table}[htb]
  \centering\small
  \caption{Table number and title are above the table}
  \label{tab:exampletable}
  \begin{tabular}{cl}
    \toprule
    Type   & Description                                       \\
    \midrule
    Hanging table & The hanging table, also known as the system table or organization table, is used to display the system structure \\
    Wire-free table & The wire-free table is generally used for equipment configuration lists, technical parameter lists, etc.   \\
    Card line table & The card line table has three types: complete table, incomplete table, and three-line table       \\
    \bottomrule
  \end{tabular}
  \note{Note: There are two types of table notes. The first is a comment on the entire table, which is placed at the bottom of the table without adding Arabic numerals,
    preceded by "Note:"; the second is a note that corresponds to some text or number in the table,
    marked in the table with a circled Arabic numeral in the upper right corner, and then noted at the bottom of the table with the same circle code}
\end{table}

The compilation of tables should be simple and clear, consistent in expression, clear and easy to understand, and the table and text should correspond to each other and be consistent in content.
When typesetting, the font size of the table should be slightly smaller, or the font should be changed, try not to paginate, and try not to cross sections.
When the table is too large and needs to be turned over, it should be noted "Continued table" above the continued table, and the table header page should be repeated.



\section{Illustrations}

Some students may have heard that "\LaTeX{} can only use eps format images", and even convert jpg format to eps.
In fact, this practice is outdated.
And every time you compile, you have to call external tools to parse eps, which slows down the compilation speed.
So we recommend using pdf format for vector graphics and jpeg or png format for bitmap.
\begin{figure}[htb]
  \centering
  \includegraphics[width=0.3\textwidth]{ustc_logo_fig.pdf}
  \caption{Figure number and title are placed below the figure}
  \label{fig:logo}
  \note{Note: The content of the figure note should not be placed in the figure title.}
\end{figure}

Regarding the side-by-side arrangement of pictures, it is recommended to use the newer \pkg{subcaption} package,
and it is not recommended to use packages such as \pkg{subfigure} or \pkg{subfig}.



\section{Algorithm Environment}

The \pkg{algorithm2e} package is used in the template to implement the algorithm environment. For specific usage of this package,
please read the official documentation of the package.

\begin{algorithm}[htb]
  \small
  \SetAlgoLined
  \KwData{this text}
  \KwResult{how to write algorithm with \LaTeX2e }

  initialization\;
  \While{not at end of this document}{
    read current\;
    \eIf{understand}{
      go to next section\;
      current section becomes this one\;
    }{
      go back to the beginning of current section\;
    }
  }
  \caption{Algorithm Example 1}
  \label{algo:algorithm1}
\end{algorithm}

Note that we can insert algorithms in the paper, but inserting large sections of code is foolish.
However, this does not prevent some students from choosing to do so. For these students, it is recommended to use the \pkg{listings} package.
