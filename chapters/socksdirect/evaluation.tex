\section{评估}
\label{socksdirect:sec:evaluation}


%\begin{figure}[htpb]
%	\centering
%	\includegraphics[width=\columnwidth]{eval/microbenchmark/fork-tput.pdf}
%	\caption{Throughput of SocksDirect with fork happening}
%	\label{socksdirect:fig:eval-fork-tput}
%\end{figure}

\sys{} 在三个组件中实现:一个用户空间库 \libipc {} 和一个带有17K行C ++代码的监控守护进程,以及一个支持零拷贝的修改过的RDMA 网卡驱动程序。
本节从以下方面评估\sys{}:

\parab {有效地为主机内套接字使用共享内存。}
对于8字节消息,\sys 实现0.3 $\mu$s RTT和每秒23~M消息的吞吐量。 对于大型消息,\sys 使用零拷贝来实现Linux的1/13延迟和26x吞吐量。

\parab {有效地使用RDMA进行主机间套接字。}
\sys 达到1.7 $\mu$s RTT,接近原始RDMA性能。
零拷贝时,一个连接会使100~Gbps链路饱和。

%\parab{Robust with number of connections.}
%The performance above can be maintained with up to 100 million connections.

%\parab{Corner-case operations does not affect long-term performance.}
%After corner-case operations such as \texttt{fork}, the performance recovers quickly.

\parab {可扩展核心数。}
随着核心数量的增加,吞吐量几乎可以线性扩展。


\parab {使用未经修改的端到端应用程序显着加速。}
例如,\sys  {}将Nginx HTTP请求延迟减少5.5倍到20倍。


\subsection{方法}
\label{socksdirect:subsec:methodology}

本节使用两个Xeon E5-2698 v3 CPU,256~GiB内存和一个Mellanox ConnectX-4网卡的服务器评估 \sys{}。服务器与Arista 7060CX-32S 交换机通过 100 Gbps 以太网接口互连 \cite {arista-7060cx}。不同于第 \ref{chapter:clicknp}、\ref{chapter:kvdirect} 章,本节仅使用可编程网卡中的商用网卡部分,且商用网卡升级到了 100 Gbps,没有使用 FPGA。服务器使用Ubuntu 16.04和Linux 4.15,将RoCEv2用于RDMA协议,每64条消息轮询一次完成队列。
每个线程都固定在CPU内核上。在收集数据之前,进行了足够的预热测试。
对于延迟,本节使用一个乒乓应用程序报告平均往返时间,误差条代表1%和99%百分位数。
对于吞吐量,一方保持发送数据而另一方不断接收数据。
本节将比较Linux,原始RDMA写原语(write verb),Rsocket~ \cite {rsockets}和LibVMA~ \cite {libvma},这是针对Mellanox 网卡优化的用户空间TCP / IP协议栈。
本节没有评估mTCP~ \cite {jeong2014mtcp},因为底层DPDK库对Mellanox ConnectX-4网卡的支持有限。由于批处理,mTCP具有比RDMA高得多的延迟,报告的吞吐量为每秒1.7~M包~~ \cite {kalia2018datacenter}。

\subsection{微基准测试}
\label{socksdirect:subsec:microbenchmark}

\subsubsection{延迟和吞吐量}



图 \ref {socksdirect:fig:eval-msgsize-intra}显示了一对发送方和接收方线程之间的主机内套接字性能。
对于8字节消息,\sys 实现0.3 $ \mu $ s往返延迟(Linux的1/35)和每秒23~M消息吞吐量(Linux的20倍)。
相比之下,一个简单的共享内存队列具有0.25 $ \mu $ s往返延迟和27~M吞吐量,表明\sys 增加了很少的开销。
RSocket具有6x延迟和1/4吞吐量的\sys  {},因为它使用网卡转发主机内数据包,这会导致PCIe延迟。
LibVMA只是将内核TCP套接字用于主机内部。
\sys  {}的单向延迟为0.15 $ \mu $ s,甚至低于内核交叉(0.2 $ \mu $ s)。基于内核的套接字需要在发送方和接收方都进行内核交叉。


由于内存复制,对于8~KiB消息,\sys 的吞吐量仅比Linux高60%,延迟低4倍。对于大小至少为16~KiB的消息,\sys 使用页面重映射来实现零拷贝。
对于1~miB消息,\sys 比Linux具有1/13延迟和26x吞吐量。
由于事件通知延迟,RSocket的延迟不稳定,在某些情况下甚至可能比Linux大。



\begin{figure*}[htbp]
	\centering
	\subfloat[单机内吞吐量。]{                    
		%\begin{minipage}{0.4\textwidth}
		\centering
		\includegraphics[width=0.5\textwidth]{eval/microbenchmark/msgsize-ipc-tput.pdf}
		\label{socksdirect:fig:eval-msgsize-ipc-tput}
		%\end{minipage}
	}
	\subfloat[单机内延迟。]{
		%\begin{minipage}{0.4\textwidth}
		\centering \includegraphics[width=0.5\textwidth]{eval/microbenchmark/msgsize-ipc-lat.pdf}
		\label{socksdirect:fig:eval-msgsize-ipc-lat}
		%\end{minipage}
	}
	
	\caption{不同消息大小下的单机内通信单核消息性能。}
	\label{socksdirect:fig:eval-msgsize-intra}
\end{figure*}


图 \ref {socksdirect:fig:eval-msgsize-inter}显示了一对线程之间的主机间套接字性能。
对于8字节消息,\sys 实现每秒18M消息吞吐量(Linux的15倍)和1.7微秒/秒的延迟(Linux的1/17)。
吞吐量和延迟接近原始RDMA写操作(如虚线所示),它没有套接字语义。
由于批处理,\sys  {}对于8字节消息的吞吐量甚至高于RDMA。
对于16到128字节的消息,LibVMA还使用批处理来实现比\sys  {}更好的吞吐量,但延迟是\sys  {}的7倍。
对于小于8~KiB的消息大小,主机间RDMA的吞吐量略低于主机内共享内存,因为环形缓冲区结构是共享的。
对于512B到8KiB消息,\sys  {}受数据包复制的限制,但由于缓冲管理开销减少,仍然比RSocket和LibVMA更快。
对于零拷贝消息($\ge$ 16 KiB),\sys  {}使网络带宽饱和,其具有所有比较工作的3.5倍吞吐量和RSocket的72%延迟。


\begin{figure*}[htbp]
	\centering
	\subfloat[跨主机吞吐量。]{
		%\begin{minipage}{0.4\textwidth}
		\centering \includegraphics[width=0.5\textwidth]{eval/microbenchmark/msgsize-network-tput.pdf}
		\label{socksdirect:fig:eval-msgsize-network-tput}
		%\end{minipage}
	}
	\subfloat[跨主机延迟。]{
		%\begin{minipage}{0.4\textwidth}
		\centering \includegraphics[width=0.5\textwidth]{eval/microbenchmark/msgsize-network-lat.pdf}
		\label{socksdirect:fig:eval-msgsize-network-lat}
		%\end{minipage}
	}
	
	\caption{不同消息大小下的跨主机通信单核消息性能。}
	\label{socksdirect:fig:eval-msgsize-inter}
\end{figure*}




\subsubsection{多核可扩放性}






\sys 实现了主机内和主机间套接字的几乎线性可扩展性。
对于主机内套接字,\sys 在16对发送器和接收器核心之间提供每秒306~M消息的吞吐量,这是Linux的40倍和RSocket的30倍。
LibVMA回退到Linux用于主机内套接字。
使用RDMA作为主机间套接字,\sys 使用批处理以16个内核实现每秒276~M个消息的吞吐量,这是本章使用的RDMA网卡的消息吞吐量的2.5倍。
由于缓冲区管理的可扩展性有限,RSocket的主机内部为24~M,主机间为33~M。
由于共享网卡队列上的锁争用,与单线程相比,LibVMA的吞吐量减少到两个线程的1/4,而三个和更多线程的1/10。
Linux吞吐量从1到7个核心线性扩展,并在环回或具有更多核心的网卡队列上出现瓶颈。


\begin{figure*}[htbp]
	\subfloat[单机内吞吐量。]{                    
		%\begin{minipage}{0.4\textwidth}
		\centering
		\includegraphics[width=0.5\textwidth]{eval/microbenchmark/corenum-IPC-tput.pdf}
		\label{socksdirect:fig:eval-cornum-ipc}
		%\end{minipage}
	}
	\subfloat[跨主机吞吐量。]{
		%\begin{minipage}{0.4\textwidth}
		\centering \includegraphics[width=0.5\textwidth]{eval/microbenchmark/corenum-network-tput.pdf}
		\label{socksdirect:fig:eval-cornum-network}
		%\end{minipage}
	}
	
	\caption{不同 CPU 核数下的 8 字节消息吞吐量。}
	\label{socksdirect:fig:eval-corenum-tput}
\end{figure*}


%The multi-thread scalability of \sys  attributes to the partitioning of states and removal of synchronization.
%We can also see that shared memory communication has 5x throughput than RDMA.% Using RDMA 网卡 for intra-host socket would meet this bottleneck and thus not scalable.

%Figure~\ref{socksdirect:fig:eval-conn-setup-tput} shows the throughput of connection creation with different number of cores. Each core can create 1.4~M new connections per second, which is 20x of Linux and 2x of mTCP~\cite{jeong2014mtcp}. The upper bound is 5.3~M connections per second, where the monitor becomes a bottleneck.

最后评估共享核心的多个线程的性能。 每个线程都需要等待轮到它来处理消息。
如图 \ref {socksdirect:fig:eval-context-switch}所示,尽管消息处理延迟几乎与活动进程的数量呈线性增长,但它仍然是Linux的1/20到1/30。


\begin{figure}[htbp]
	%\centering
	%\includegraphics[width=\textwidth]{eval/microbenchmark/conn-setup-tput.pdf}
	%
	%\caption{Connection creation throughput with number of cores.}
	%\label{socksdirect:fig:eval-conn-setup-tput}
	
	%\begin{minipage}{0.4\textwidth}
	\centering \includegraphics[width=0.5\textwidth]{eval/microbenchmark/sharecore-lat.pdf}
	
	\caption{多进程共享 CPU 核的消息处理延迟。}
	\label{socksdirect:fig:eval-context-switch}
	%\end{minipage}
\end{figure}


%Finally, we benchmark the throughput and latency after \texttt{fork} and other corner-case operations. Initially, there is only one pair of sender and receiver. At time $T_0$, receiver forks, and the parent process keeps receiving. At time $T_1$, the child process begins to receives takes over the socket. At time $T_2$, sender forks, and only the parent sends. At time $T_3$, the child sender also starts sending. We find that both throughput and latency resume to initial maximal performance within 1~ms after each event.
