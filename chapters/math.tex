% !TeX root = ../main.tex

\chapter{Mathematics}

\section{Numbers and Units}

The package \pkg{siunitx} provides better support for numbers and units:
\begin{itemize}
  \item \num{12345.67890}
  \item \num{1+-2i}
  \item \num{.3e45}
  \item \num{1.654 x 2.34 x 3.430}
  \item \si{kg.m.s^{-1}}
  \item \si{\micro\meter} $\si{\micro\meter}$
  \item \si{\ohm} $\si{\ohm}$
  \item \numlist{10;20}
  \item \numlist{10;20;30}
  \item \SIlist{0.13;0.67;0.80}{\milli\metre}
  \item \numrange{10}{20}
  \item \SIrange{10}{20}{\degreeCelsius}
\end{itemize}



\section{Mathematical Symbols and Formulas}

\LaTeX{} typesets mathematical formulas and symbols by default according to American conventions,
but the "Manual of Style" requires that mathematical symbols be executed according to "GB 3102.11--1993",
which differs from the \LaTeX{} convention.
This template configures mathematical symbols based on the \pkg{unicode-math} package to comply with the national standard.

Note that the \pkg{unicode-math} package is \emph{not} compatible with the \pkg{amsfonts}, \pkg{amssymb}, \pkg{bm},
\pkg{mathrsfs}, \pkg{upgreek} and other packages.
This template has been processed, and users can directly use the commands of these packages, such as \cs{bm}, \cs{mathscr},
\cs{upGamma}.

The usage of mathematical symbols in this template is somewhat different from the traditional \LaTeX{}:
\begin{itemize}
  \item Mathematical constants and special functions use upright type,
    such as the circular ratio $\symup{\pi}$, $\symup{\Gamma}$ function.
    The \cs{symup} command provided by the \pkg{unicode-math} package should be used to convert to upright type,
    such as \verb|\symup{\pi}|.
  \item Vectors and matrices are bold italic, and the \cs{symbf} command should be used,
    such as \verb|\symbf{u}|, \verb|\symbf{A}|.
  \item The finite increment symbol $\increment$ (U+2206) should use the \cs{increment} command.
  \item The differential symbol $\dif$ uses upright type, and this template provides the \cs{dif} command.
\end{itemize}

In addition, the template also provides some commands for convenience:
\begin{itemize}
  \item Constant $\upe$: \verb|\upe|
  \item Negative unit $\upi$: \verb|\upi|
  \item Circular ratio $\uppi$: \verb|\uppi|
  \item $\argmax$: \verb|\argmax|
  \item $\argmin$: \verb|\argmin|
\end{itemize}

For more usage of mathematical symbols, refer to the usage instructions and symbol list of the \pkg{unicode-math} package
\pkg{unimath-symbols}.

When editing mathematical formulas, it is best to avoid using font commands directly,
but should define some semantic commands to replace font commands,
which makes input simpler, makes \LaTeX{} code more readable,
and also facilitates the unified modification of the format as needed, such as:
\begin{itemize}
  \item Vector $\vec{x}$: \verb|\renewcommand\vec{\symbf}|
  \item Matrix $\mat{A}$: \verb|\newcommand\mat{\symbf}|
  \item Tensor $\ts{T}$: \verb|\newcommand\ts{\symbfsf}|
\end{itemize}

More examples:
\begin{equation}
  \upe^{\upi\uppi} + 1 = 0
\end{equation}
\begin{equation}
  \frac{\dif^2 u}{\dif t^2} = \int f(x) \dif x
\end{equation}
\begin{equation}
  \argmin_x f(x)
\end{equation}
\begin{equation}
  \mat{A} \vec{x} = \lambda \vec{x}
\end{equation}



\section{Theorems and Proofs}

The example file uses the \pkg{amsthm} package to configure environments such as theorems, lemmas, and proofs.
Users can also use the \pkg{ntheorem} package.

The content you provided is already in English and in academic style. As per your instructions, I will not make any changes to it.

The content you provided is already in English and in academic style. As per your instructions, I will keep it as-is. Here is the content:

\begin{theorem}
  Suppose $\{f_n\}$ is a sequence of measurable functions such that
  $f_n(x) \to f(x)$ a.e. $x$, as $n$ tends to infinity.
  If $|f_n(x)| \le g(x)$, where $g$ is integrable, then
  \begin{equation}
    \int |f_n - f| \to 0 \qquad \text{as } n \to \infty,
  \end{equation}
  and consequently
  \begin{equation}
    \int f_n \to \int f \qquad \text{as } n \to \infty.
  \end{equation}
\end{theorem}

\begin{proof}
  Trivial.
\end{proof}

\newtheorem*{axiomofchoice}{Axiom of choice}
\begin{axiomofchoice}
  Suppose $E$ is a set and ${E_\alpha}$ is a collection of
  non-empty subsets of $E$. Then there is a function $\alpha
  \mapsto x_\alpha$ (a ``choice function'') such that
  \begin{equation}
    x_\alpha \in E_\alpha,\qquad \text{for all }\alpha.
  \end{equation}
\end{axiomofchoice}

\newtheorem{observation}{Observation}
\begin{observation}
  Suppose a partially ordered set $P$ has the property
  that every chain has an upper bound in $P$. Then the
  set $P$ contains at least one maximal element.
\end{observation}
\begin{proof}[A concise proof]
  Obvious.
\end{proof}
