% !TeX root = ../main.tex

\begin{acknowledgements}

我要感谢我的母校中国科学技术⼤学和微软亚洲研究院能给我宝贵的学习机会,让我能在联合培养博士期间接触到世界领先的可编程网卡实验平台和数据中心应用场景。

我要感谢中国科学技术⼤学的导师陈恩红教授。
从本科四年级开始,直到硕博连读毕业的六年里,陈恩红老师一直支持我在微软的联合培养实习,帮助我确定了研究方向和博士课题。
读博期间,陈老师帮助我确定培养计划,资助我参加国际学术会议,推荐我申请微软学者奖、国家奖学金等诸多奖励,还帮助我修改开题报告和毕业论文。
无论是生活还是科研,陈老师都竭尽所能给予我⽀持和帮助,让我免除后顾之忧,集中注意⼒于科研中的学术问题。
我能取得一些小小的学术成果,不仅有赖于陈⽼师在大方向上的指导,也跟陈老师在背后默默的⽀持和帮助是密不可分的。
衷心地感谢陈恩红老师对我的支持和帮助。

我要感谢我在微软亚洲研究院的导师,首席研究员张霖涛博士。
相处的三年里,张霖涛老师带领我走进系统研究的大门,不仅教会了我计算机系统的知识和思维方式,而且锻炼了我独立思考、发现问题和主持研究的能力。
张霖涛老师指导我完成了第二个研究项目 KV-Direct,在键值存储领域若干可能的创新点里,选定了加速内存数据结构访问这个最能突出可编程网卡作用的创新点。
在我和阮震元同学合作实现的过程中,他帮助我们提炼总结系统设计与优化技巧,从头到尾修改论文、讲稿,并发表在系统领域的顶级学术会议上,让我在博士中期有较好的科研成果。
随后,张霖涛老师带领我开阔视野,培养对系统的大局观,并帮助我招聘实习生来合作实现我的创新。
他既给我足够的空间让我独立思考、自由探索,又能敏锐地理解我的想法并给出深刻的建议,提醒我不要陷入技术细节而忘记系统的目标,还给我在知名教授面前锻炼演讲和听取反馈的宝贵机会。
张霖涛老师指导我理清了博士期间研究的主线,认识到自己所做研究更深刻的内涵、更广阔的外延以及与高影响力工作的差距。
张霖涛老师带我在微软总部进行了第一次美国之旅,平时经常给我分享系统研究、职场和人生的经验,是我的良师益友。
衷心地感谢张霖涛老师对我的指导和帮助。

我要感谢我在微软亚洲研究院的前导师,前资深研究员谭焜博⼠。
谭焜老师是我的科研启蒙导师,不仅教会了我计算机⽹络的知识和思维⽅式,⽽且教会了我科研 ``分析型思考'' 的⽅法论和做学问 ``去伪存真'' 的态度。
谭焜老师确立了网络研究组在数据中心领域的研究方向,搭建了世界领先的数据中心网络和可编程网卡实验平台。
谭焜老师手把手指导我完成了第一个研究项目 ClickNP。他提出了用可编程网卡加速网络功能这个学术问题,确定了高级语言编程的基本框架和技术路线,帮助我撰写论文,并发表在网络领域的顶级学术会议上,让我在科研上有一个较高的起点。
谭焜老师一方面让我给不同领域的研究员讲解以锻炼大局观,另一方面注重细节,在组会上集体讨论代码风格、实验数据和讲稿字句。
在我思维过于发散时,他及时让我收敛得出结论,让我能持续高效产出。
ClickNP 项目完成后,我在 FPGA 编程语言、系统和网络三个方向之间纠结时,谭焜老师指导我加入系统组,打开了新世界的大门。
谭焜老师还给我分享了很多对研究的哲学思考,亦是我的良师益友。
衷⼼地感谢谭焜⽼师对我的指导和帮助。

我要感谢微软雷德蒙研究院的张永光首席研究员,
谢谢他能在我⼤四的时候给我提供进⼊中国科学技术⼤学和微软亚洲研究院联合培养的机会。

我要感谢我科研项⽬的合作者们。他们除了我的导师以外,还有微软亚洲研究院的副院长周礼栋博士,网络组研究员白巍博士,系统组任晶磊博士、陈亮博士,网络组资深研究员熊勇强博士、研究员程鹏博士,湖南⼤学的陈果副教授,阿⾥巴巴计算平台的张建松博士,腾讯⽹络平台的罗腊咏博士,南京⼤学的王晓亮副教授,腾讯网络平台的陆元伟博士,北京航天航空⼤学的博⼠⽣肖⽂聪,美国麻省理工学院的博⼠⽣阮震元,美国华盛顿大学的博士生崔天一,美国密歇根州立大学的博士生左格非,中国科学技术大学与微软亚洲研究员联合培养的博士生王子博。感谢他们在我的科研项⽬中给予的⽀持和帮助。

我还要感谢在中国科学技术⼤学和微软亚洲研究院遇到的⽼师,同学和朋
友们,谢谢你们对我的⽀持和⿎励,帮助我渡过⼀个又⼀个难关。衷⼼地谢谢你
们。

最后,我要感谢我的⽗母和家⼈,是他们在背后默默地⽀持着我,让我有⼀
个强有⼒的后盾。感谢我的⽗母⼆⼗多年的养育之恩,以及其他家⼈的帮助和⽀
持。没有你们,我不可能做出现在的成绩。你们是最伟⼤的。谢谢你们的付出。

\end{acknowledgements}
