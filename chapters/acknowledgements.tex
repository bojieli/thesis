% !TeX root = ../main.tex

\begin{acknowledgements}

我要感谢我的母校中国科学技术⼤学和微软亚洲研究院能给我宝贵的学习机会,让我能在联合培养博士期间接触到世界领先的可编程网卡实验平台和数据中心应用场景,在世界级团队中开展数据中心可编程网卡的先驱研究。

我要感谢中国科学技术⼤学的导师陈恩红教授。
从本科四年级开始的六年里,陈恩红老师一直支持我在微软的联合培养实习,帮助我确定了研究方向和博士课题。
读博期间,陈老师帮助我确定培养计划,资助我参加国际学术会议,推荐我申请微软学者奖、国家奖学金等诸多奖励,还帮助我修改开题报告和毕业论文。
无论是生活还是科研,陈老师都竭尽所能给予我⽀持和帮助,让我免除后顾之忧,集中注意⼒于科研中的学术问题。
我能取得一些小小的学术成果,不仅有赖于陈⽼师在大方向上的指导,也跟陈老师在背后默默的⽀持和帮助是密不可分的。
衷心地感谢陈恩红老师对我的支持和帮助。

我要感谢我在微软亚洲研究院的导师,首席研究员张霖涛博士。
相处的三年里,张霖涛老师带领我走进系统研究的大门,不仅教会了我计算机系统的知识和思维方式,而且锻炼了我独立思考、发现问题和主持研究的能力。
张霖涛老师指导我完成了第二个研究项目 KV-Direct,在键值存储领域若干可能的创新点里,选定了加速内存数据结构访问这个最能突出可编程网卡作用的创新点。
在我和阮震元同学合作实现的过程中,他帮助我们提炼总结系统设计与优化技巧,从头到尾修改论文、讲稿,并发表在系统领域的顶级学术会议上,让我在博士中期有较好的科研成果。
随后,张霖涛老师给我足够的空间让我独立思考、自由探索,带领我开阔视野,培养对系统的大局观,并帮助我招聘实习生来合作实现我的创新,研究了几个新的课题,论文被 SIGCOMM 等会议接收。
不管是组内还是组外的报告,张霖涛老师总是能敏锐地理解并提出深刻的问题。他给我在知名教授面前讲故事和听取反馈的宝贵机会,提醒我不要陷入技术细节而忘记听众的背景和系统的大局。
张霖涛老师指导我理清了博士期间研究的主线,认识到自己所做研究更深刻的内涵、更广阔的外延以及与高影响力工作的差距。
张霖涛老师带我在微软总部进行了第一次美国之旅,平时经常给我分享系统研究、职场和人生的经验,是我的良师益友。
衷心地感谢张霖涛老师对我的指导和帮助。

我要感谢我在微软亚洲研究院的前导师,前资深研究员谭焜博⼠。
谭焜老师是我的科研启蒙导师,不仅教会了我计算机⽹络的知识和思维⽅式,⽽且教会了我科研 ``分析型思考'' 的⽅法论和做学问 ``去伪存真'' 的态度。
谭焜老师确立了网络研究组在数据中心领域的研究方向,搭建了世界领先的数据中心网络和可编程网卡实验平台。
谭焜老师手把手指导我完成了第一个研究项目 ClickNP。他提出了用可编程网卡加速网络功能这个学术问题,确定了高级语言编程的基本框架和技术路线,帮助我撰写论文,并发表在网络领域的顶级学术会议上,让我在科研上有一个较高的起点。
谭焜老师一方面让我给不同领域的研究员讲解以锻炼大局观,另一方面注重细节,在组会上讨论代码风格、实验数据和讲稿字句。
在我思维过于发散时,他及时让我收敛得出结论,让我能持续高效产出。
ClickNP 项目完成后,我在 FPGA 编程和系统、网络方向之间纠结时,谭焜老师指导我定位在系统领域,专注于能产生实际影响的项目。
谭焜老师还给我分享了很多对研究的哲学思考,亦是我的良师益友。
衷⼼地感谢谭焜⽼师对我的指导和帮助。


我要感谢与我合作论文的老师和同学们。他们除了我的导师以外,还有微软亚洲研究院的研究员和实习生同学。在 ClickNP 项目中,我要感谢研究员罗腊咏博士在 FPGA 编程框架方面的早期探索,上海交通大学彭燕庆同学、中国科学技术大学罗人千同学合作开发编译器、网络元件和应用,资深研究员徐宁仪博士、资深研究员熊勇强博士、研究员程鹏博士的讨论与帮助,以及北京航空航天大学的贺同同学在 FPGA 与 CPU 间通信管道的开发。
在 KV-Direct 项目中,我要感谢共同第一作者、中国科学技术大学阮震元同学与我一起设计和实现系统,北京航空航天大学肖文聪同学撰写引言,中国科学技术大学陆元伟同学的讨论与帮助,资深研究员熊勇强博士和首席硬件工程师 Andrew Putnam 博士的讨论与硬件实验环境的支持。
在 SocksDirect 项目中,我要感谢共同第一作者、中国科学技术大学崔天一同学与我一起设计和实现系统,研究员白巍博士撰写引言、梳理论文逻辑,中国科学技术大学王子博同学实现了第一版系统的原型。
在 TOMS 项目中,我要感谢中国科学技术大学左格非同学与我一起设计和实现系统,研究员白巍博士与我讨论修改了论文的大部分内容。
衷心感谢所有博士论文合作者对我的指导与帮助,没有你们,我是不可能做出这些成果的。

除了以上论文合作者外,我还要感谢微软亚洲研究院副院长周礼栋博士,系统组张建松博士、任晶磊博士、陈亮博士,湖南⼤学的陈果副教授,南京⼤学的王晓亮副教授在科研项⽬中给予我的指导和帮助。特别是陈果副教授和张建松博士与我多次合作论文。我要感谢首席研究员张永光博士,给我进入中国科学技术⼤学和微软亚洲研究院联合培养的机会。我要感谢在学术会议和面试中给予我宝贵指导的老师、专家和同学。

我要感谢在中国科学技术⼤学和微软亚洲研究院遇到的⽼师、同学和朋友们。特别是微软亚洲研究院 ``饭团'' 的同学,谢谢你们陪伴我度过丰富多彩的博士生活。感谢微软亚洲研究院学术合作部的同事们在我的博士培养过程与实习生活中的帮助。也要感谢中国科学技术大学 Linux 用户协会的张焕杰老师和各位小伙伴们,让我在本科期间得以开发和运维各种网络服务,锻炼了计算机系统的技术能力,培养了我对云计算的兴趣。还要感谢中学的肖世康同学,教我做网站;区块链领域的朋友们,培养了我对定制化硬件的兴趣。我要特别感谢我的女朋友,我们不仅因为科研相识,一起合作论文,她还非常支持我的学术研究和职业发展规划。我的女朋友让我在生活上变得更成熟,也使我们在一起的生活充满了快乐。她就像一束光,照到的地方就充满了光亮。

最后,我要感谢我的⽗母、爷爷奶奶和家⼈。是你们在背后默默地⽀持着我,让我有⼀个强有⼒的后盾。感谢我的⽗母和爷爷奶奶⼆⼗多年的养育之恩,以及其他家⼈的帮助和⽀持。没有你们,我不可能做出现在的成绩。你们是最伟⼤的,谢谢你们的付出。
\end{acknowledgements}
