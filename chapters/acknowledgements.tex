\begin{acknowledgements}

I would like to express my gratitude to my alma mater, the University of Science and Technology of China, and Microsoft Research Asia for providing me with valuable learning opportunities. These allowed me to engage with the world-leading programmable network card experimental platform and data center application scenarios during my joint doctoral training, and to conduct cutting-edge research in the field of data center systems.

I am grateful to my supervisor at the University of Science and Technology of China, Professor Chen Enhong. Over the past six years, starting from my senior year of undergraduate studies, Professor Chen has consistently supported my joint training internship at Microsoft, helping me to determine my research direction and doctoral topic. During my doctoral studies, Professor Chen assisted me in formulating my training plan, funded my participation in international academic conferences, recommended me for numerous awards such as the Microsoft Scholar Award and National Scholarship, and helped me revise my proposal and dissertation. Whether in life or in scientific research, Professor Chen has done everything possible to support and assist me, allowing me to focus on academic issues in research. The modest academic achievements I have made are not only due to Professor Chen's guidance on the overall direction, but also inseparable from his silent support and assistance. I sincerely thank Professor Chen Enhong for his support and help.

I would like to thank my supervisor at Microsoft Research Asia, Dr. Zhang Lintao, a principal researcher. During the three years we worked together, Dr. Zhang led me into the field of system research, not only teaching me the knowledge and thinking methods of computer systems, but also training my ability to think independently, identify problems, and lead research. Dr. Zhang guided me to complete my second research project, KV-Direct, and among several possible innovation points in the field of key-value storage, he selected the acceleration of memory data structure access, which best highlights the role of programmable network cards. During the implementation process in collaboration with my classmate Ruan Zhenyuan, he helped us refine and summarize system design and optimization techniques, revised the paper and presentation from beginning to end, and published it in a top academic conference in the system field, allowing me to have good research results in the middle of my doctorate. Subsequently, Dr. Zhang gave me enough space to think independently and explore freely, broadened my horizons, cultivated my overall view of the system, and helped me recruit interns to cooperate in implementing my innovations, researched several new topics, and had papers accepted by conferences like SIGCOMM. Whether it's internal or external reports, Dr. Zhang is always able to understand and raise profound questions keenly. He gave me valuable opportunities to tell stories and receive feedback in front of renowned professors, reminding me not to get lost in technical details and forget the audience's background and the overall picture of the system. Dr. Zhang guided me to clarify the main line of research during my doctorate, understand the deeper connotations and broader extensions of my research, and the gap with high-impact work. Dr. Zhang took me on my first trip to the United States at Microsoft's headquarters, and often shared with me his experiences in system research, the workplace, and life. He is my good teacher and helpful friend. I sincerely thank Dr. Zhang Lintao for his guidance and help.

I would like to thank my former supervisor at Microsoft Research Asia, former senior researcher Dr. Tan Kun. Dr. Tan is my scientific research enlightenment mentor, who not only taught me the knowledge and thinking methods of computer networks, but also taught me the methodology of "analytical thinking" in scientific research and the attitude of "discarding the false and retaining the true" in scholarship. Dr. Tan established the research direction of the network research group in the field of data centers and built a world-leading data center network and programmable network card experimental platform. Dr. Tan guided me to complete my first research project, ClickNP, hand in hand. He proposed the academic problem of accelerating network functions with programmable network cards, determined the basic framework and technical route of high-level language programming, helped me write papers, and published them in top academic conferences in the network field, giving me a high starting point in research. Dr. Tan allowed me to explain to researchers in different fields to exercise my overall view, and on the other hand, he paid attention to details, discussing code style, experimental data, and presentation sentences at group meetings. When my thinking was too divergent, he timely let me converge to draw conclusions, allowing me to continuously produce efficiently. After the completion of the ClickNP project, when I was torn between FPGA programming and the direction of systems and networks, Dr. Tan guided me to position myself in the field of systems and focus on projects that can have a real impact. Dr. Tan also shared with me many philosophical thoughts on research and is also my good teacher and helpful friend. I sincerely thank Dr. Tan Kun for his guidance and help.

I would like to thank the teachers and classmates who co-authored papers with me. In addition to my supervisors, they also include researchers and intern students at Microsoft Research Asia. In the ClickNP project, I would like to thank Dr. Luo Layong, a researcher, for his early exploration in the field of FPGA programming framework, Peng Yanqing from Shanghai Jiaotong University, Luo Renqian from the University of Science and Technology of China for their cooperation in developing compilers, network elements, and applications, Dr. Xu Ningyi, a senior researcher, Dr. Xiong Yongqiang, a senior researcher, Dr. Cheng Peng, a researcher, for their discussions and help, and He Tong from Beijing University of Aeronautics and Astronautics for the development of communication pipelines between FPGA and CPU. In the KV-Direct project, I would like to thank Ruan Zhenyuan, a co-first author from the University of Science and Technology of China, for designing and implementing the system with me, Xiao Wencong from Beijing University of Aeronautics and Astronautics for writing the introduction, Lu Yuanwei from the University of Science and Technology of China for his discussions and help, Dr. Xiong Yongqiang, a senior researcher, and Dr. Andrew Putnam, a chief hardware engineer, for their discussions and support for the hardware experimental environment. In the SocksDirect project, I would like to thank Cui Tianyi, a co-first author from the University of Science and Technology of China, for designing and implementing the system with me, Dr. Bai Wei, a researcher, for writing the introduction and sorting out the logic of the paper, and Wang Zibo from the University of Science and Technology of China for implementing the first version of the system prototype. In the TOMS project, I would like to thank Zuo Gefei from the University of Science and Technology of China for designing and implementing the system with me, and Dr. Bai Wei, a researcher, for discussing and revising most of the paper content with me. I would like to thank Dr. Ren Jinglei, a researcher from the system group at Microsoft Research Asia, Dr. Chen Liang, Wang Yang from the University of Electronic Science and Technology, Taekyung Heo from KAIST, and Lu Yuanwei, Xiao Wencong, Ruan Zhenyuan, Cui Tianyi, Li Yishuai, Cao Shijie and other classmates for their cooperation and help in the yet-to-be-published research projects. I would also like to give special thanks to Dr. Chen Guo, a researcher at Microsoft Research Asia, and Dr. Zhang Jiansong for their multiple paper collaborations. Dr. Chen Guo's FUSO was my first paper in my academic career as the fourth author, and his rigorous academic attitude has benefited me for life. Dr. Zhang Jiansong guided me in FPGA development, and I was fortunate to participate in the Feniks FPGA OS and SSD acceleration projects proposed by Dr. Zhang Jiansong. I sincerely thank all the co-authors of my doctoral thesis for their guidance and help. Without you, I could not have achieved these results.

In addition to the co-authors of this thesis, I would also like to thank Dr. Lidong Zhou, the deputy director of Microsoft Research Asia, Dr. Yunxin Liu and Dr. Chen Zhang from the Systems Group, Dr. Ran Shu and Dr. Zhixiong Niu from the Networking Group, and Associate Professor Xiaoliang Wang from Nanjing University for their guidance and assistance in my research projects. I am grateful to Andrew Putnam, Tanj Bennett, Derek Chiou, Qi Luo, Daniel Firestone from Microsoft's headquarters in the United States, and Wei Cui, Wenqiang Wang and other colleagues from Microsoft Research Asia for their support and help. I would like to thank Dr. Yongguang Zhang, a chief researcher, for giving me the opportunity to be jointly trained by the University of Science and Technology of China and Microsoft Research Asia. I appreciate the valuable comments from the anonymous reviewers of the conference papers and the doctoral thesis review committee. I am grateful to the teachers, experts, and classmates who gave me valuable guidance in academic conferences and interviews.

I would like to thank the teachers, classmates, and friends I met at the University of Science and Technology of China and Microsoft Research Asia. Especially the "Rice Ball" students who are the core of the joint doctoral training at Microsoft Research Asia, thank you for accompanying me through the colorful doctoral life. I am grateful to the colleagues of the Academic Cooperation Department of Microsoft Research Asia for their help in my doctoral training and internship life. I would like to thank Mr. Huanjie Zhang and all the partners of the Linux User Association of the University of Science and Technology of China, for giving me the opportunity to develop and maintain various network services during my undergraduate years, which honed my technical ability in computer systems and the psychological quality of handling faults, and cultivated my interest in cloud computing. I am grateful to my high school classmate Xiaoshikang for teaching me how to make websites, and friends in the blockchain field for cultivating my interest in customized hardware. I would like to give special thanks to my girlfriend. Not only did we get to know each other through research and co-authored papers, but she also strongly supports my academic research and career development planning. My girlfriend has made me more mature in life and filled our life together with happiness. She is like a beam of light, illuminating everything with brightness.

Finally, I would like to thank my parents, grandparents, and family. It is you who silently support me from behind, giving me a strong backing. I am grateful for the kindness of my parents and grandparents for more than twenty years, as well as the help and support from other family members. Without you, I could not have achieved what I have today. You are the greatest, thank you for your dedication.
\end{acknowledgements}
