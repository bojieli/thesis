\chapter{系统架构}

基于可重构硬件的智能网卡。需要讨论以智能网卡为中心的服务器软硬件架构。基本思想:可编程网卡是服务器与外界之间通信的“网关”,也是服务器内硬件设备、虚拟机间通信的“枢纽”,把 hypervisor 和 OS kernel 中需要高性能的数据平面卸载到可编程网卡。

画一个数据中心架构图,包括存储、网关等节点,计算节点(包括 hypervisor 和客户虚拟机)。

\section{网络、存储节点的硬件加速}

位置:存储、网关等节点。CPU bypass(控制面数据面分离,数据面 offload,控制面仍在 CPU 上)s

\subsection{网络数据包处理}

直接处理网络数据包,数据面无需经过 CPU(ClickNP NF offload)。

\subsection{访问内存和闪存中的数据结构}

直接访问远程硬件资源,而无需经过远程机器的 CPU (KV-Direct内存数据结构的服务器端,还可以访问闪存(SSD)记录日志后直接返回 ACK,缩短处理延迟)。

\section{计算节点的硬件加速}

位置:计算节点(客户虚拟机所在的服务器)。虚拟机监控器(hypervisor):硬件的一虚多、多虚一;操作系统:高层抽象。

\subsection{一虚多}

可编程网卡把主机内的硬件资源虚拟化成多个逻辑资源,实现外部机器和本地虚拟机的多路复用(ClickNP 硬件网卡虚拟化为多个租户的 VPC,SocksDirect 容器网络,即 vSwitch data-plane offload)。

\subsection{多虚一}

可编程网卡把数据中心内物理上分散的资源虚拟化成一个逻辑资源(ClickNP VPC 和 SocksDirect 容器网络,KV-Direct 分布式存储的客户端,还可以做 storage 和 memory 的 disaggregation)。

\subsection{高层抽象}

OS kernel 给应用程序提供的抽象可以重构为(控制面)协调和管理(仍在内核或用户态 daemon) + (数据面)用户态 library 负责高层抽象 + (数据面)可编程网卡负责多路复用、调度唤醒和可靠传输等低层语义,需要思考数据面上软硬件的接口(SocksDirect)。

\section{可编程网卡的硬件编程}

可编程网卡承担这么多功能,需要高度灵活性。

\subsection{FPGA 高级语言编程}

需要适合流式处理的模块化 FPGA 高级语言编程(ClickNP 编程框架)。

\subsection{硬件中的并发事务处理}

与软件不同。简化:事务的依赖性可以事先判断。KV-Direct 有依赖请求的并发执行框架。