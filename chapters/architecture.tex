\chapter{系统架构}

需要讨论以可编程网卡为中心的服务器软硬件架构。基本思想:可编程网卡是服务器与外界之间通信的“网关”,也是服务器内硬件设备、虚拟机间通信的“枢纽”,把 hypervisor 和 OS kernel 中需要高性能的数据平面卸载到可编程网卡。

1. 可编程网卡把主机内的硬件资源虚拟化成多个逻辑资源,实现外部机器和本地虚拟机的多路复用(一虚多,ClickNP网络虚拟化为例)。

2. 可编程网卡把数据中心内物理上分散的资源虚拟化成一个逻辑资源(多虚一,KV-Direct内存数据结构的客户端为例,还可以做 memory disaggregation)。

3. 绕过远程 CPU 直接访问远程硬件资源(KV-Direct内存数据结构的服务器端为例,还可以访问 SSD、GPU)。

4. OS kernel 给应用程序提供的抽象,可以重构为(控制面)协调和管理(仍在内核或用户态 daemon) + (数据面)用户态 library 负责高层抽象 + (数据面)可编程网卡负责多路复用、调度唤醒和可靠传输等低层语义(SocksDirect)。

5. 可编程网卡承担这么多功能,需要高度灵活性,因此需要适合流式处理的模块化 FPGA 高级语言编程(ClickNP)。