\chapter{系统架构}

需要讨论以可编程网卡为中心的服务器软硬件架构。基本思想:可编程网卡是服务器与外界之间通信的“网关”,也是服务器内硬件设备、虚拟机间通信的“枢纽”,把 hypervisor 和 OS kernel 中需要高性能的数据平面卸载到可编程网卡。

1. CPU bypass

1.1. 直接处理网络数据包,数据面无需经过 CPU(ClickNP NF offload)。

1.2. 直接访问远程硬件资源,而无需经过远程机器的 CPU (KV-Direct内存数据结构的服务器端,还可以访问 SSD、GPU)。

1.3. 可编程网卡负责多路复用、调度唤醒和可靠传输(RDMA,SocksDirect)。

2. 虚拟化

2.1. 一虚多:可编程网卡把主机内的硬件资源虚拟化成多个逻辑资源,实现外部机器和本地虚拟机的多路复用(ClickNP 硬件网卡虚拟化为多个网卡,即 vSwitch offload)。

2.2. 多虚一:可编程网卡把数据中心内物理上分散的资源虚拟化成一个逻辑资源(ClickNP 虚拟网络VPC,KV-Direct内存数据结构的客户端,还可以做 storage 和 memory 的 disaggregation)。

2.3. 高层抽象:OS kernel 给应用程序提供的抽象:可以重构为(控制面)协调和管理(仍在内核或用户态 daemon) + (数据面)用户态 library 负责高层抽象 + (数据面)可编程网卡负责多路复用、调度唤醒和可靠传输等低层语义(SocksDirect)。

3. 可编程网卡承担这么多功能,需要高度灵活性,因此需要适合流式处理的模块化 FPGA 高级语言编程(ClickNP)。