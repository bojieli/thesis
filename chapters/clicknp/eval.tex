%!TEX root=main.tex
\section{评估}
\label{clicknp:sec:eval}

\subsection{测试床与方法}

我们在16台Dell R720服务器的测试台中评估\name。
对于每个FPGA板,两个以太网端口都连接到架顶式(ToR)Dell S6000交换机\cite {dells6000}。
所有\name NF都在Windows Server 2012 R2上运行。
我们将\name 与其他最先进的软件NF进行比较。
对于在Linux上运行的那些NF,我们使用内核版本为3.10的CentOS 7.2。
在我们的测试中,我们使用PktGen发包工具以不同的速率生成具有不同数据包大小的测试流量(64B 数据包,最高吞吐量为 56.4 Mpps)。
为了测量NF处理延迟,我们在每个测试数据包中嵌入了一个生成时间戳。
当数据包通过NF时,它们被循环回到PktCap抓包工具,该PktCap与PktGen位于同一FPGA中。
然后我们可以通过从数据包的接收时间中减去生成时间戳来确定延迟。
由PktGen和PktCap引起的延迟通过直接环回(无NF)进行预校准,并从我们的数据中删除。

\egg{
To measure NF processing latency, we forward packets to an \textit{Echo} server using the second port. 
The Echo server runs an \textit{Echo} function in its FPGA, which simply bounces all packets back to the source. 
%
Then, we can compare the difference between the timestamps of a packet 
when it first arrives and when the processed packet is bounced back, to obtain the latency 
with nanosecond accuracy.
%
The delay induced by the Echo server was pre-calibrated and removed from our data.
In our test, we use PktGen to generate testing traffic, which can 
generate packets at up to 56.4 Mpps (64B packets).
}

\egg{
Ethernet ports of FPGAs and servers are connected to a Dell S6000 40GbE switch 
FPGAs are installed on Windows Server 2012 R2.
CPU-based benchmarks are performed on 

We use TrafficGen application to benchmark throughput and latency.
TrafficGen application generates a given traffic pattern or replays a trace, embeds timestamp in packet payload and sends to tor\_out.
Applications forward packets from tor\_in to nic\_out. When TrafficGen receives packet from nic\_in, the timestamp is extracted from payload to measure latency.
The wire loopback RTT of TrafficGen is 0.85$\mu$s for 64B packets and 1.39$\mu$s for 1504B packets, which is the system error in all latency numbers we present.
}

\begin{figure*}[t!]
		\centering
		\subfloat[]{
			\includegraphics[width=0.5\textwidth]{eval/fw_1}
		}
	%	\subfloat[]{
	%		\includegraphics[width=0.225\textwidth]{eval/fw_2}
	%	}
		\subfloat[]{
			\includegraphics[width=0.5\textwidth]{eval/fw_3}
		}
	\hspace{1pt}
		\subfloat[]{
			\includegraphics[width=0.5\textwidth]{eval/fw_4}
		}
		\subfloat[]{
			\includegraphics[width=0.5\textwidth]{eval/fw_5}
		}
		
		\caption{Firewalls. Error bars represents the $5^{th}$ and $95^{th}$ percentile. (a) and (b) Packet size is 64B.}
		
		\label{clicknp:fig:firewall}
\end{figure*}

\begin{figure*}[t!]
		\centering
		\subfloat[]{
			\includegraphics[width=0.5\textwidth]{eval/ipsec_1}
		}
		\subfloat[]{
			\includegraphics[width=0.5\textwidth]{eval/ipsec_2}
		}
		
		\caption{IPSec gateway. }
		
		\label{clicknp:fig:IPSec}
\end{figure*}

\subsection{吞吐量和延迟}

%In this subsection, we benchmark the throughput and latency of \name\ NFs.

\textbf {A2. OpenFlow防火墙。}
在本实验中,我们将OpenFlow防火墙与Linux防火墙以及Click + DPDK~ \cite {barbette2015fast}进行比较。
对于Linux,我们使用IPSet来处理完全匹配规则,同时将IPTable用于通配符规则。
作为参考,我们还包括Dell S6000交换机的性能,该交换机具有有限的防火墙功能并支持1.7K通配符规则。
值得注意的是,最初的Click + DPDK~ \cite {barbette2015fast}不支持接收端缩放(RSS)。
在这项工作中,我们已经修复了这个问题并发现当使用4个内核时,Click + DPDK已经达到了最佳性能。
但对于Linux,我们使用尽可能多的内核(由于RSS限制,最多8个内核)以获得最佳性能。

图 \ref {clicknp:fig:firewall}(a)显示了具有不同数量的通配符规则的不同防火墙的数据包处理速率。
数据包大小为64B。
我们可以看到\name 和S6000都可以达到56.4~Mpps的最大速度。
Click + DPDK可以达到约18~Mpps。
由于Click使用静态分类树来实现通配符匹配,因此处理速度不会随插入规则的数量而变化。
Linux IPTables具有2.67~Mpps的低处理速度,并且随着规则数量的增加速度降低。这是因为IPTables为通配符规则执行线性匹配。

图 \ref {clicknp:fig:firewall}(b)显示了使用小数据包(64B)和8K规则的不同负载下的处理延迟。
由于每个防火墙具有显着不同的容量,因此将负载因子标准化为每个系统的最大处理速度。
在所有负载水平下,FPGA(\name{})和ASIC(S6000)解决方案具有 $\mu$s 级别的延迟(\name{} 为1.23 $\mu$s,S6000为0.62 $\mu$s),方差非常小( ClickNP为1.26 $\mu$s,对于S6000为95%(百分位数)为0.63 $\mu$s)。
但是,软件解决方案具有更大的延迟,并且方差也更大。
例如,使用Click + DPDK,当负载很高时,延迟可能高达$ 50 \mu{}s $。
图 \ref {clicknp:fig:firewall}(c)显示了不同数据包大小和8K规则的处理延迟。
使用软件解决方案,延迟随数据包大小的增加而增加,这主要是由于要复制的内存较大。
相比之下,FPGA和ASIC保持相同的延迟,而与数据包大小无关。
在所有实验中,\name 防火墙的CPU使用率非常低(单个 CPU 核的$ <5 %$)。

最后,图 \ref {clicknp:fig:firewall}(d)显示了已有8K规则时的规则插入延迟。 Click的静态分类树需要事先了解所有规则,而生成8K规则的树需要一分钟。
IPTables规则插入需要12 ms,这与表中现有规则的数量成比例。
Dell S6000中的规则插入需要83.7 $\mu$s。
对于\name{},在HashTCAM表中插入一个规则需要6.3至9.5$\mu$s用于2至3次PCIe往返,而SRAM TCAM表平均需要44.9 $\mu$s来更新13个查找表。
\name 数据平面吞吐量在规则插入期间不会降低。
我们得出结论,\name{} 防火墙在数据包处理方面具有与ASIC类似的性能,但具有灵活性和可重构性。


\textbf {A3. IPSec 网关。}
我们比较IPSec 网关和StrongSwan~ \cite {strongswan},使用相同的密码套件AES-256-CTR和SHA1。
我们设置了一个IPSec隧道,图 \ref {clicknp:fig:IPSec}(a)显示了不同数据包大小的吞吐量。
对于所有规模,IPSecGW实现了线路速率,即64B数据包为28.8 Gbps,1500B数据包为37.8 Gbps。
然而,StrongSwan最多只能达到628 Mbps,随着数据包变小,吞吐量也会降低。
这是因为尺寸越小,需要处理的数据包数量就越多,
因此系统需要计算更多的SHA1签名。
图 \ref {clicknp:fig:IPSec}(b)显示了不同负载因子下的延迟。 同样,IPSecGW产生的恒定延迟为13 $\mu$s,
但是StrongSwan会产生更大的延迟和更高的方差,最长可达5 ms!

\begin{figure*}[t!]
	\centering
	
	\subfloat[]{
		\includegraphics[width=0.5\textwidth]{eval/l4_2}
	}
	\subfloat[]{
		\includegraphics[width=0.5\textwidth]{eval/l4_1}
	}

	\subfloat[]{
		\includegraphics[width=0.5\textwidth]{eval/l4_3}
	}

	\caption{L4 Load Balancer.}
	\label{clicknp:fig:l4}

\end{figure*}


\textbf {A4. L4负载均衡器。}
我们将L4LB与Linux虚拟服务器(LVS)进行比较 \cite {lvs}。
为了对系统进行压力测试,我们使用64B数据包生成大量并发UDP流,目标是一个虚拟IP(VIP)。
图 \ref {clicknp:fig:l4}(a)显示了具有不同并发流数的处理速率。
当并发流量小于8K时,L4LB达到51.2Mpps的线路速率。
但是,当并发流的数量变大时,处理速率开始下降。
这是因为L4LB中的流缓存未命中。
当流缓存中缺少流时,L4LB必须访问板载DDR内存,
这会导致性能下降。
当流量太多时,例如32M,缓存未命中占主导地位且对于大多数数据包而言,
L4LB需要一次访问DDR内存。因此处理速度降低到11Mpps。
在任何情况下,LVS的处理速率都很低。
由于LVS将VIP关联到仅一个CPU核心,因此其处理速率必须达到200Kpps。

图 \ref {clicknp:fig:l4}(b)显示了不同负载条件下的延迟。
在这个实验中,我们将并发流的数量修复为100万。
我们可以看到L4LB实现了4美元/秒的非常低的延迟。
然而,LVS会导致约50美元的延迟。
当提供的负载高于100Kpps时,这种延迟会迅速上升,超过LVS的容量。

最后,图 \ref {clicknp:fig:l4}(c)比较L4LB和LVS接受新流的能力。
在这个实验中,我们指示PktGen生成尽可能多的单包微流。
我们可以看到L4LB每秒可以接受高达10M的新流量。
由于单个PCIe插槽每秒可以传输16.5M的传输,因此瓶颈仍然是DDR访问。
我们的 \textit {DIPAlloc} 元素只是以循环方式分配DIP。
对于复杂的分配算法,\textit {DIPAlloc} 的CPU核心将成为瓶颈,并且可以通过在更多CPU核心上复制 \textit {DIPAlloc} 元素来提高性能。
对于LVS,由于数据包处理能力有限,它每秒最多只能接受75K新流量。


\textbf {A5. pFabric 验证。}
\name 也是网络研究的好工具。
由于灵活性和高性能,我们可以快速制作最新研究的原型并将其应用于真实环境。
例如,我们可以使用\name 轻松实现pFabric调度程序\cite {pfabric},并将其应用于我们的测试平台。
在本实验中,我们修改了一个软件TCP流生成器\cite {mqecn},以便在数据包有效负载中放置流优先级,即流的总大小。
我们根据\cite {pfabric}中的数据挖掘工作负载生成流,并使用\textit {RateLimit}元素将限制出口端口进一步设置为10~Gbps。
我们应用pFabric根据流优先级调度出口缓冲区中的流量。
图 \ref {clicknp:fig:pfabric}显示了pFabric,具有Droptail队列的TCP的平均流完成时间(FCT)和理想值。
该实验验证了pFabric在这种简单的场景中实现了接近理想的FCT。

\begin{figure}[h!]
	\centering
	\includegraphics[width=0.6\textwidth]{eval/pfabric}
	
	\caption{Validation of pFabric.}
	\label{clicknp:fig:pfabric}
	
\end{figure}



\textbf{A6. RoCE 传输协议。}

\textbf{A7. MP-RDMA 拥塞控制协议。}

\textbf{A8. HTTPS RSA 加速。}

\textbf{A9. 端口扫描检测。}

\textbf{A10. 正则表达式匹配。}

\textbf{A11. 神经网络推理。}

\egg{
\textbf{Stateful L4 load balancer.} We compare our ClickNP implementation with Linux Virtual Server (LVS) \cite{lvs} using both real-world traffic trace on a L4 load balancer \cite{gandhi2014duet} and synthetic traces to simulate adversary scenarios.
The real-world trace contains 1.3M flows collected in two hours with 26Gbps average throughput and 45KB median flow size.
The first adversary trace is round-robin scheduling packets from a large number of infinite UDP flows with 64B packet size to stress the load balancer under high concurrency.
The second adversary trace is a lot of tiny UDP flows to test how many new connections the load balancer can process per second.

Figure \ref{clicknp:fig:l4} shows that ClickNP L4 load balancer has 50\approx500x lower latency than LVS on real-world trace, supports 32M concurrent flows and able to accept \approx10M new flows per second.
This performance is comparable to high-end hardware load balancers \cite{f5loadbalance}, while ours have low cost and high flexibility.
Figure \ref{clicknp:fig:l4} also shows the importance of SRAM cache and pipelined DRAM access.
Our board can perform at most 13.6M DRAM random reads per second, and our new flow allocation rate gets near this limit thanks to pipelined DRAM access.
}

\egg{
First we test the performance under real-world data center traffic. We adopted the same traffic trace as DUET\cite{}, and picked the first ten minutes as testing data. Figure \ref{clicknp:fig:l4} shows the four experiments we performed on the L4 load balancer. We first use TCP traffic to evaluate the flow completion time (FCT) under different throughputs. Then we use UDP traffic to evaluate the latency. 

Next we tested the performance with adversary traffic to thest the worst case performance. We fixed every flow to be one 1504 byte packet, and test the throughput under different flow numbers, and latency under different new flow rates.
}

\subsection{资源利用率}

\begin{table}[t!]
	\centering
	
	\caption{ClickNP 网络功能汇总。}
	\label{clicknp:tab:applications}
	\begin{tabular}{l|r|r|r|r}
		\toprule
		Network Function & LoC$^\dagger$ & \#Elements & LE & BRAM \\
		\midrule
		\egg{
			Pkt generator & 13 & 6 & 16\% & 12\% \\
			Pkt capture & 12 & 10 & 8\% & 5\% \\
			OpenFlow firewall & 23 & 7 & 32\% & 54\% \\
			IPSec gateway & 37 & 10 & 35\% & 74\% \\
			L4 load balancer & 42 & 13 & 36\% & 38\% \\
			pFabric scheduler & 23 & 7 & 11\% & 15\% \\
		}
		Pkt generator & 665 & 6 & 16\% & 12\% \\
		Pkt capture & 250 & 11 & 8\% & 5\% \\
		OpenFlow firewall & 538 & 7 & 32\% & 54\% \\
		IPSec gateway & 695 & 10 & 35\% & 74\% \\
		L4 load balancer & 860 & 13 & 36\% & 38\% \\
		pFabric scheduler & 584 & 7 & 11\% & 15\% \\
		\bottomrule
		\multicolumn{5}{l}{$^\dagger$ Total line of code of all element declarations and} \\
		\multicolumn{5}{l}{configuration files.}
	\end{tabular}
	
\end{table}

\egg{
	To evaluate ClickNP's area cost overhead compared to hand-written HDL, we implemented several ClickNP elements resembling network functions in NetFPGA 10G \cite{netfpga} reference router and Openflow switch projects.
	Table \ref{clicknp:tab:netfpga} compares ClickNP's relative area cost over NetFPGA using Vivado HLS 2015.4 and Altera OpenCL 15.1.
	Most ClickNP implementations show less than 100\% logic overhead and less than 30\% BRAM overhead. For tiny elements (\eg IP checksum), a fixed element overhead dominates.
}

\begin{table}[t!]
	\centering
	
	\caption{相比 NetFPGA 的面积开销。}
	\label{clicknp:tab:netfpga}
	\begin{tabular}{l|r|r|r}
		\toprule
		\multirow{2}{2.2cm}{NetFPGA Function} & LUTs & Registers & BRAMs \\
		& Min / Max & Min / Max & Min / Max \\
		\midrule
		Input arbiter  & 2.1x / 3.4x & 1.8x / 2.8x & 0.9x / 1.3x \\
		Output queue   & 1.4x / 2.0x & 2.0x / 3.2x & 0.9x / 1.2x \\
		Header parser  & 0.9x / 3.2x & 2.1x / 3.2x & N/A \\
		Openflow table & 0.9x / 1.6x & 1.6x / 2.3x & 1.1x / 1.2x \\
		\midrule
		\midrule
		IP checksum    & 4.3x / 12.1x & 9.7x / 32.5x & N/A \\
		Encap          & 0.9x / 5.2x & 1.1x / 10.3x & N/A \\
		\bottomrule
	\end{tabular}
	
\end{table}


在本小节中,我们评估\name NFs的资源利用率。
表 \ref {clicknp:tab:applications}总结了结果。
除了使用大多数BRAM来保存编码书的IPSec网关之外,所有其他NF仅使用中等资源(5 至 50 %)。
仍有空间容纳更复杂的NF。

接下来,我们研究了\name 的细粒度模块化的开销。
由于每个元素都将生成逻辑块边界并仅使用FIFO缓冲区与其他块进行通信,因此应该存在开销。
为了衡量这种开销,我们创建了一个只将数据从一个输入端口传递到输出端口的简单``\textit{空}''元素。
此\textit {空}元素的资源利用率应该很好地捕获模块化的开销。
不同的HLS工具可能使用不同数量的资源,但都很低,最小值为0.15%,最大值为0.4%。
因此,我们得出结论:由于模块化,名称产生的开销很小。

最后,我们要研究\name 与手写HDL相比生成的RTL代码的效率。
为此,我们使用NetFPGA \cite {netfpga}作为参考。
我们提取NetFPGA中的关键模块,这些模块由经验丰富的Verilog程序员进行了优化,
并在\ name中实现具有相同功能的对应元素。
我们使用不同的HLS工具作为后端来比较这两种实现之间的相对面积成本。
结果总结在表 \ref {clicknp:tab:netfpga}中。
由于不同的工具可能具有不同的面积成本,因此我们记录最大值和最小值。
我们可以看到,与手工优化代码相比,自动生成的HDL代码使用更多区域。
然而,差异并不是很大。
对于复杂模块(如表格顶部所示),相对面积成本小于2倍。
对于微小模块(如表格底部所示),相对面积成本看起来更大,但绝对资源使用量很小。
这是因为所有的HLS工具都会产生一个固定的开销,占据微小模块的面积成本。


总之,\name 可以为FPGA生成高效的RTL,只需要适度的面积成本,可以构建实用的NF。
展望未来,FPGA技术仍在迅速发展。 例如,Altera的下一代FPGA Arria 10的容量将比我们目前使用的芯片多2.5倍。
因此,我们认为HLS的面积成本将来不会受到关注。


\egg{
Figure \ref{clicknp:tab:applications} shows Logic Elements (LE) and BRAM footprint of aforementioned ClickNP applications.
Catapult shell and OpenCL runtime take 30\% LEs and 18\% BRAMs in addition to ClickNP applications.
}


\egg{
Original data:
Function	LUTs			Registers			BRAMs		
NetFPGA	AOCL	VHLS	NetFPGA	AOCL	VHLS	NetFPGA	AOCL	VHLS
Input arbiter	2048	4346	6998	4553	12843	8348	30	40	26
Output queue	3824	5280	7542	6479	13409	21005	22.5	26	20
Header parser	626	2012	567	698	6095	817	N/A		
Openflow table	3923	6322	3535	4438	10171	7140	33.5	40	38
IP checksum	207	2511	884	162	5269	1576	N/A		
Encap	684	3536	636	821	8493	871	N/A	
}

\egg{
\subsection{Network Benchmark Suite}

\subsubsection{Configurable Flow Generator}

\begin{lstlisting}
Rand(0,1,2) -> MetaGen(1,1) -> RevParser(1,1) -> IPChecksum(1,1) -> TCPChecksum(1,1) -> NVGRE_Encap(1,1) -> tor_out
\end{lstlisting}

\subsubsection{Packet Trace Replay}

\begin{figure}[h!]
	\centering
	\includegraphics[width=0.6\columnwidth]{image/logo}
	\vspace{-0.15in}
	\caption{Traffic Replay Throughput, x: packet size, y: Gbps, lines: ClickNP, CPU}
	\vspace{-0.15in}
	\label{clicknp:fig:TrafficReplayPerformance}
	%    
\end{figure}

\subsubsection{Traffic Monitor}

\begin{lstlisting}
tor_in -> Receiver -> Drop (1,0)
\end{lstlisting}

latency, throughput, sequence number check

\subsection{Network Virtualization}

\subsubsection{NVGRE Tunneling}

\begin{figure}[h!]
	\centering
	\includegraphics[width=0.6\columnwidth]{image/logo}
	\vspace{-0.15in}
	\caption{Tunnel Encap + Decap Performance, x: packet size, y: Gbps, lines: ClickNP, Hyper-V}
	\vspace{-0.15in}
	\label{clicknp:fig:NVGREPerformance}
	%    
\end{figure}

\subsubsection{Per-VM Metering and Rate Limiting}

\subsection{Security}

\subsubsection{Network-layer Firewall}

\begin{lstlisting}
Parser :: parser(1,2)
DropPolicer :: action(2,2)
tor_in -> parser[1] -> [1]action[1] -> nic_out
parser[2] -> ExtractFiveTuple(1,1) -> Hashtable @(1,1) -> [2]action[2] -> Drop (1,0)
\end{lstlisting}

\begin{figure}[h!]
	\centering
	\includegraphics[width=0.6\columnwidth]{image/logo}
	\vspace{-0.15in}
	\caption{Network-layer Firewall Performance, x: \#rules, y: pps, lines: ClickNP, Linux iptables}
	\vspace{-0.15in}
	\label{clicknp:fig:FirewallPerformance}
	%    
\end{figure}

\subsubsection{Application-layer Firewall}

\begin{lstlisting}
Parser :: parser(1,2)
DropPolicer :: action(2,2)
tor_in -> parser[1] -> [1]action[1] -> nic_out
parser[2] -> ExtractFiveTuple(1,1) -> RegexMatch @(1,1) -> [2]action[2] -> Drop (1,0)
\end{lstlisting}

\begin{figure}[h!]
	\centering
	\includegraphics[width=0.6\columnwidth]{image/logo}
	\vspace{-0.15in}
	\caption{Application-layer Firewall Performance, x: \#rules, y: Gbps, lines: ClickNP, snort \cite{roesch1999snort}}
	\vspace{-0.15in}
	\label{clicknp:fig:WAF_Performance}
	%    
\end{figure}

\subsubsection{DDoS Detection with Bitmap Sketch}

\begin{lstlisting}
Parser :: parser(1,2)
tor_in -> parser[1] -> nic_out
parser[2] -> ExtractVMID(1,1) -> HashTable @(1,1) -> ExtractSrcIP(1,1) -> BitmapSketch @(1,1) -> Drop (1,0)
\end{lstlisting}

OpenSketch \cite{yu2013software}

\begin{figure}[h!]
	\centering
	\includegraphics[width=0.6\columnwidth]{image/logo}
	\vspace{-0.15in}
	\caption{Sketch Accuracy, x: \#real flows, y: relative error (\%) with error bar}
	\vspace{-0.15in}
	\label{clicknp:fig:SketchAccuracy}
	%    
\end{figure}


\subsection{Scheduling}

\subsubsection{PIAS Tagging}

PIAS \cite{bai2014pias}

\subsubsection{Generic Priority Queue}

pFabric \cite{alizadeh2013pfabric}

\subsubsection{Packet Pacing}

Silo \cite{jang2015silo}

\begin{figure}[h!]
	\centering
	\includegraphics[width=0.6\columnwidth]{image/logo}
	\vspace{-0.15in}
	\caption{CDF of pacing inaccuracy due to buffer overflow, x: percentile, y: time shift, lines: buffer sizes}
	\vspace{-0.15in}
	\label{clicknp:fig:PacingAccuracy}
	%    
\end{figure}

\subsubsection{Timestamping}

TIMELY \cite{mittal2015timely} measures RTT as the signal for congestion control. It requires both receive and send timestamping feature and hardware-generated ACK of recent NICs. Furthermore, since ACKs have already been sent by NIC hardware, OS networking stack needs modifications to avoid generating duplicate ACKs.

With ClickNP we can do the same latency measurement without using a NIC with timestamping feature, and does not require modification to the OS networking stack, as long as the FPGAs at sender and receiver have same clock frequency. The idea is to subtract the latency spent in receiver-side OS networking stack. Upon reception of a data packet, we subtract the timestamp and expects the OS to echo back the timestamp in the ACK packet. When the ACK packet is sent, we add the timestamp, as if we had shifted the initial timestamp by OS processing latency. Pseudo code:

\begin{lstlisting}
.element SendTimestamp {
    .state { uint timestamp = 0; }
    .handler {
        if (input_ready) {
            if (!is_ack_packet())
                set_tsval (timestamp);
            else
                set_tsecr (get_tsecr() + timestamp);
        }
        timestamp ++;
    }
}
.element RecvTimestamp {
    .state { uint timestamp = 0; }
    .handler {
        if (input_ready) {
            if (!is_ack_packet())
                set_tsval (get_tsval() - timestamp);
            else
                set_tsval (timestamp);
        }
    }
}
nic_in -> Parser (1,2) -> SendTimestamp (2,1) -> tor_out
tor_in -> Parser (1,2) -> RecvTimestamp (2,1) -> nic_out
\end{lstlisting}

\begin{figure}[h!]
	\centering
	\includegraphics[width=0.6\columnwidth]{image/logo}
	\vspace{-0.15in}
	\caption{CDF of measured RTT, x: percentile, y: RTT, lines: hardware ACK (ideal), remove OS latency (close to ideal), not remove OS latency (poor)}
	\vspace{-0.15in}
	\label{clicknp:fig:TimestampAccuracy}
	%    
\end{figure}
}

\egg{
Figure~\ref{clicknp:fig:trafficgen} shows the throughput of \name\ traffic generator and capture. We can see that the generator can
generate packets at line-rate of 40 Gbps. 
When packet size is small ($<256B$), the measured throughput is slightly less than 40 Gbps. We confirm this is due to a
bug in Ethernet MAC in the shell, which may drop a few flits when packet rate is high. 
Figure~\ref{clicknp:fig:trafficgen}(b) shows the capture performance in packet per second.  

\begin{figure}[t!]
	\centering
	\subfloat[] {
	\includegraphics[width=0.225\textwidth]{eval/trafficgen}
	}
	\subfloat[] {
	\includegraphics[width=0.225\textwidth]{eval/dump}
	}
	
	\caption{(a) Traffic generator. (b) Traffic capture.}
	\label{clicknp:fig:trafficgen}
\end{figure}
}
\egg{
In this section we evaluate throughput and latency of applications presented in section \ref{clicknp:sec:application}. All these application can achieve line-rate throughput and microsecond-scale latency for reasonable traffic patterns. Latency is averaged over $1000$ rounds, and error bars mark the $5\%$ and $95\%$ percentile.

\textbf{Traffic generator.} We use our traffic generator to send packets to NIC, and use Windows Performance Monitor to measure the throughput. As shown in Figure \ref{clicknp:fig:trafficgen}, our generator is able to generate packets at 40Gbps line rate when the packet size is not less than 256 bytes. The small gap for 64-byte packets is due to lack of back-pressure in 40GbE MAC of FPGA shell.

\textbf{Traffic capture.} We evaluate the throughput of dumper and results are presented in \ref{clicknp:fig:dump}.
Each host kernel runs on a separate core and traffic is split evenly to kernels by flow tuple hash.
With four host elements receiving the packets, we can approximately achieve the maximum speed of the PCIe channel.
When we only need the flow tuple and timestamp, ClickNP can easily capture line rate with two cores.
}

\egg{
\textbf{IPSec gateway.} We use StrongSwan \cite{strongswan} with AES256CTR + SHA1 IKEv2 cipher suite as baseline.
After setting up IPSec keys and nonces in ClickNP elements via signal, we send UDP packets in a single IPSec tunnel to test throughput and latency on encryption path.
StrongSwan leverages only one CPU core due to tunnel message sequencing.
As shown in Figure \ref{clicknp:fig:IPSec}, \textit{Reservo} to exploit packet-level parallelism for SHA-1 element shows 40x performance gain compared to unoptimized version. }

\egg{
\textbf{OpenFlow firewall.} We compare our firewall with Click + DPDK \cite{barbette2015fast} on 4 cores, and Linux iptables (for wildcard match) / ipset \cite{ipset} (for exact match) with RSS on 8 cores.
We also include Dell S6000, a high-end commodity switch, as a reference.

We optimized FromDPDKDevice element in Click to enable receiving packets on multiple cores, while the bottleneck becomes Mellanox polling-mode driver at 18 Mpps.
Click uses a radix tree for exact IP lookup and a classification tree for wildcard flow tuple lookup, so table lookup is not bottleneck of Click.
Linux ipset uses hash table for exact lookup. Linux iptables match wildcard rules linearly, which is the source of low throughput and high latency.
Dell S6000 supports 1.7K 5-tuple wildcard match rules.
For all packet sizes and number of rules, ClickNP offers line rate throughput and latency comparable to commodity ASIC.

Additionally, for 8K wildcard rules, Click takes minutes to generate the classification tree and cannot offer live rule update. ClickNP can perform 350K live rule updates per second while the data plane keeps line forwarding rate. Dell S6000 can perform 12K live rule updates per second.

16K exact, 8 K wild.
}

\egg{
All results are presented in Figure \ref{clicknp:fig:firewall}. The first two subfloats are to show how packet size and rule numbers influence throughput. In the first subfloat, the packet size is fixed at 64 byte. And in the second subfloat, the numbers of rules is set to be 64k(exact)/8k(wild).

The last two subfloats are about latency. In \ref{clicknp:}, we fixed the packet size to be ?? and evaluated the latency with different loads. In \ref{clicknp:}, we change the packet size to see how the latency changes in maximum load scenario. 

From all these figures we can see that ClickNP approximates the performance of Dell S6000, and has a ???x boost on the CPU solution.
}
