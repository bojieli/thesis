%!TEX root=main.tex
\section{背景}
\label{clicknp:sec:background}

\subsection{软件网络功能的性能挑战}

软件NF具有极大的灵活性和可扩展性。
早期研究主要集中在基于软件的数据包转发 \cite {routebricks,Egi:2008:THP:1544012.1544032}。
他们表明,多核x86 CPU可以以每服务器10Gbps的速度转发数据包,并且可以通过集群更多服务器来扩展容量。
最近,许多系统被设计用于实现各种类型的NF \cite {comb,Greenhalgh:2009:FPR:1517480.1517484,martins2014clickos}。
同样,所有这些系统都利用CPU中的多核并行性来实现每台机器接近10Gbps的吞吐量,并在需要更高容量时向外扩展以使用更多机器。
Ananta \cite {ananta}是一个部署在Microsoft数据中心的软件负载均衡器,用于提供云规模的负载均衡服务。
虽然软件NF可以扩展以提供更多容量,但这样做会在CAPEX和OPEX中增加相当大的成本 \cite {ananta,duet}。

为了加速软件包处理,以前的工作已经提出使用GPU \cite {packetshader},专用网络处理器(NP)\cite {cavium,netronome} 和硬件交换机 \cite {duet}。
GPU主要用于图形处理,最近扩展到具有海量数据并行性的其他应用程序。 GPU更适合批量操作。 PacketShader \cite {packetshader},表明使用GPU可以实现40Gbps的分组交换速度。
但是,批量操作会导致高延迟。
例如,\cite {packetshader}中报告的转发延迟约为$200 \mu{}s$,比 \name{} 大两个数量级。
与GPU相比,FPGA更灵活,可以重新配置以捕获数据和流水线并行,这两者在NF中非常常见。
网络处理器专门用于处理网络流量并且具有许多硬连线(hard-wired)的网络加速器。
NP-Click \cite{shah2004np} 在网络处理器上实现了 Click 编程框架。
相比之下,FPGA是一种通用计算平台。
除了NF之外,FPGA在数据中心还有许多其他应用,使其更具吸引力
大规模部署 \cite {putnam2014reconfigurable}。
硬件交换机功能有限,其应用受到很大限制 \cite {duet}。

与CPU或GPU相比,FPGA通常具有更低的时钟频率和更小的存储器带宽。
例如,FPGA的典型时钟频率约为200MHz,比CPU慢一个数量级(2至3~GHz)。
同样,FPGA的单块存储器或外部DRAM的带宽通常为2至10~GBps,而内存带宽约为Intel XEON CPU的40~GBps,GPU为100~GBps。
但是,CPU或GPU只有有限的内核,这限制了并行性。 FPGA内置了大量的并行性。
现代FPGA可能拥有数百万个LE,数百个K位寄存器,数十个M位BRAM和数千个DSP模块。从理论上讲,它们中的每一个都可以并行工作。
因此,FPGA芯片内部可能会同时运行数千个并行的``\textit {核}''。
虽然单个BRAM的带宽可能有限,但如果我们并行访问数千个BRAM,则总内存带宽可以是多TBps!
因此,为了实现高性能,程序员必须充分利用这种大规模的并行性。

传统上,FPGA使用诸如Verilog和VHDL之类的HDL进行编程。
这些语言水平太低,难以学习,编程也很复杂。
因此,大型软件程序员社区已经远离FPGA多年了~\cite {bacon2013fpga}。
为了简化这一点,许多高级综合(HLS)工具/系统已经在工业界和学术界开发,试图将高级语言(主要是C)的程序转换为HDL。
但是,正如我们将在下一小节中所示,它们都不适合网络功能处理,这是本工作的重点。

\subsection{基于 FPGA 的网络功能编程}

我们的目标是利用FPGA加速构建一个多功能,高性能的网络功能平台。这样的平台应满足以下要求。

\textbf {灵活性。} 平台应该\textit {完全使用高级语言编程。}
开发人员使用高级抽象和熟悉的工具编程,并具有类似的编程经验,就像在多核处理器上编程一样。
我们相信这是使大多数软件程序员可以使用FPGA的必要条件。

\textbf {模块化。} 我们应该支持\textit {模块化架构}进行数据包处理。以前关于虚拟化NF的经验表明,正确的模块化架构可以很好地捕获数据包处理中的许多常见功能~\cite {kohler2000click,martins2014clickos},使它们易于在各种NF中重用。

\textbf {高性能和低延迟。} 数据中心的NF应该以40 / 100~Gbps的线路速率处理大量数据包,具有超低延迟。以前的工作已经显示~\cite {rollback-mb},即使NF添加的几百微秒的延迟也会对服务体验产生负面影响。

\textbf {支持联合CPU / FPGA数据包处理。} FPGA 不是灵丹妙药。
正如第 2 章所讨论的,并非所有任务都适用于FPGA。例如,自然顺序的算法和具有非常大的内存占用和低局部性的处理应该在CPU中处理得更好。
此外,FPGA具有严格的区域约束。
这意味着您无法将任意大的逻辑放入芯片中。
在没有数据平面中断的情况下动态交换FPGA配置非常困难,因为重新配置时间可能需要几秒到几分钟,具体取决于FPGA的大小。
因此,我们应该支持CPU和FPGA之间的细粒度处理分离。这需要CPU和FPGA之间的高性能通信。


现有的FPGA高级编程工具都不能满足上述所有要求。
大多数HLS工具,例如 Vivado HLS~\cite{vivado},只是HDL工具链的辅助工具。
这些工具不是直接将程序编译成FPGA映像,而是仅生成硬件模块,即IP核,必须手动嵌入在HDL项目和连接到其它HDL模块,这是大多数软件程序员无法完成的任务。

但是,Altera OpenCL可以直接将OpenCL程序编译为FPGA~ \cite {aoc}。
但是,OpenCL编程模型直接源自GPU编程,并且不是用于数据包处理的模块化。
此外,OpenCL不支持CPU和FPGA之间的联合数据包处理:
首先,主机程序和FPGA内核之间的通信必须始终通过板载DDR内存。 这增加了非平凡的延迟,并且还导致板载内存成为瓶颈。
其次,OpenCL内核函数需要宿主机上的软件程序显式\textit {调用}。
在内核终止之前,主机程序无法控制内核行为,例如设置新参数,也不能读取任何内核状态。
但是NF面临着连续的数据包流,应该始终在运行。

FPGA是一项成熟的技术,最近已经部署用于加速数据中心服务,包括网络功能 \cite {putnam2014reconfigurable,smartnic,rubow2010chimpp,lavasani2012compiling}。
众所周知,FPGA的可编程性很低,有丰富的前人工作提供了高级编程抽象 \cite {bluespec,auerbach2010lime,bacon2013fpga,singh2011implementing,bachrach2012chisel,wester2015transformation}。
Gorilla \cite {lavasani2012compiling} 为FPGA上的数据包交换提出了一种特定于域的高级语言。
Cliff \cite{kulkarni2004mapping}、CUSP \cite{schelle2005cusp} 和 Chimpp \cite {rubow2010chimpp} 将Click模型引入硬件描述语言来开发模块化路由器。
\name 沿此方向工作,与以前的工作互补。
\name 致力于数据中心的NF,通过提供高度灵活的模块化架构和利用商业HLS工具解决可编程性问题。

Click2NetFPGA~\cite {Click2NetFPGA}通过直接将Click模块化路由器〜\ cite {kohler2000click}程序编译到FPGA中来提供模块化架构。
然而,\cite {Click2NetFPGA}的性能比我们在本文中报告的要低得多(两个数量级),因为它们的系统设计存在几个瓶颈(例如,内存和数据包I / O),它们也是 错过几个重要的优化以确保完全流水线处理(如\S \ref {clicknp:sec:optimization}中所述)。
此外,\cite {Click2NetFPGA}不支持FPGA / CPU联合处理,因此无法在数据平面运行时更新配置或读取状态。



在下面,我们将介绍\name{},一种新颖的FPGA加速网络功能平台,满足上述四个要求。

\egg{
Today's data centers rely on a wide range of network functions to implement network virtualization, ensure security (e.g. firewalls and intrusion detection/prevention systems), perform measurements and improve performance (e.g. traffic scheduling). As data centers are moving towards 40 Gbps bandwidth at end hosts, where the line-rate is 60 M packets per second for minimum-sized packets, higher throughput requirement is imposed upon network processors. Furthermore, as data center services are evolving rapidly, programmability becomes indispensable for network processors. However, existing network processors has a large mismatch to the performance and programmability requirements.

\subsection{Architectures for Network Processors}

Network processors based on general-purpose CPUs such as ClickOS \cite{martins2014clickos} enjoy good programmablity, modularity and composabilty, but the packet forwarding performance of a single core could not keep up with 10 Gbps line rate for minimum-sized packets, even before any network function is plugged in. Because CPU instructions are executed one-by-one and have low parallelism, packet processing performance would drop further as more network functions are added. If a CPU-based network processor is added bump-in-the-wire, there will be 10s of microseconds additional end-to-end latency \cite{martins2014clickos} which is one magnitude higher than the switching fabric. In network virtualization scenario, if packet encapsulation and decapsulation is done at end hosts, as in the case of virtual switch, NIC offloading mechanisms including Large Send Offload (LSO) and Large Receive Offload (LRO) have to be disabled, which has a huge impact on TCP performance \cite{yoshino2008performance}.

ASICs are known to be high-performance, but the network functions are fixed. Commodity switching ASICs typically have a pipeline of network functions \cite{broadcomethernet}, where each function can be configured via registers and a match table based on TCAM or memory. Some ASICs provide flexible OpenFlow-like match-action tables \cite{broadcomopenflow}, but the packet parser is fixed (we could not support new packet header and shim layer formats), actions are not extensible and the order of network functions in the pipeline is not reconfigurable.

GPUs are widely used as co-processors for computing-intensive tasks, but its SIMD (Single-Instruction Multiple-Data) programming model does not fit network processing, where different types of packets may take various execution flows. The high power consumption, high latency of batch processing and inability to receive and send network packets without CPU intervention are also factors that render GPU-based network processor infeasible in data centers.

Fortunately, reconfigurable hardware is an architecture that provides both programmability, high performance and power efficiency for certain workloads. FPGA (field programmable gate arrays) is the most prominent example of reconfigurable hardware. FPGAs can implement arbitrary logic function and utilize distributed on-chip registers and SRAM to exploit bit-level and task-level parallelism, therefore stream processing pipelines would not ``hit the memory wall'' as in Von Neumann architecture \cite{bacon2013fpga}. FPGA has shown potential in accelerating many workloads in cloud \cite{putnam2014reconfigurable}. Moreover, Moore's law is still working in FPGA industry, because the fabrication technology of FPGA is currently several generations behind the CPU industry [citation required].

\subsection{FPGA Programming Challenge}

Despite FPGA's potential in network processing, the programmablity of FPGA is traditionally provided by hardware description languages (HDL) such as Verilog, which requires hardware knowledge and are much harder to program and debug than higher-level languages such as C/C++. Thus, existing FPGA-based network processors such as NetFPGA \cite{lockwood2007netfpga} are hard to program for software engineers.

Many works, e.g. OpenFlow \cite{mckeown2008openflow}, P4 \cite{bosshart2014p4} and SDNet \cite{xilinxsdnet}, provides the programmability by abstracting a set of primitives in network processing and defining a high-level programming language to compose the primitives. This direction has proved effective, but the programmability is limited to a set of pre-defined actions, which could not keep pace with rapid development of data center network functions. Our work strive to make the primitives extensible for software engineers.

Fortunately, several frameworks have been proposed to provide abstractions for generic FPGA programming. Examples of such works include Xilinx Vivado HLS (High Level Synthesis) \cite{feist2012vivado} based on C/C++, Altera SDK for OpenCL \cite{czajkowski2012opencl} based on C-like OpenCL and IBM Lime \cite{auerbach2010lime} based on Java.

However, FPGA has a completely different architecture than general-purpose CPUs. For software programmers that bear Von Neumann model in mind, the compilers may generate surprisingly poor hardware logic for reasonable code in high-level language. For example, Click2NetFPGA \cite{Click2NetFPGA} uses LLVM and HLS tools to compile optimized Click C++ code into HDL, but the resulting FPGA-based router can only process 178 K pps (packets per second) for 98B packets, and 215 Mbps for large packets, which is 30 -- 50x slower than a CPU core in ClickOS \cite{martins2014clickos}. The bottleneck for small packets is the IP header checking stage \cite{Click2NetFPGA} because this stage is not fully pipelined; the bottleneck for large packets is the byte-wide shared memory \cite{Click2NetFPGA}, indicating a shared-memory design suitable for Von Neumann model would yield poor performance on FPGA.

FPGA has millions of logic gates with 10x slower clock rate than CPU, thousands of distributed fast SRAMs each with only KB capacity, and a large DRAM with 10x lower throughput than DRAMs in CPU architecture. Consequently, exploiting both spatial and temporal parallelism is crucial to unleashing the performance of FPGA. In network stream processing, most operations are independent of each other and therefore can be either parallelized (spatial) or pipelined (temporal), so that each stage of the pipeline can process different packets in parallel.

\subsection{Design Goals}
\label{clicknp:subsec:designgoals}

We highlight several design goals for our ClickNP framework to enable software engineers to write efficient network applications.

\smalltitle{Modularity.} Modularity is one key feature that improves parallelism, since modules do not have shared state and can run in parallel by nature. Borrowing the concepts from Click modular router \cite{kohler2000click}, \textit{elements} are basic building blocks of network functions. Elements run asynchronously and are connected via uni-directional \textit{channels}. The network processing pipeline is a data flow graph of elements and channels, starting from Ethernet receivers and ending at Ethernet transmitters.

\smalltitle{Line-rate throughput.} To allow efficient processing of packet content, an Ethernet packet is split into 32-byte \textit{flits} before feeding into elements. In the worst case, when 69-byte packets are received back-to-back, the line rate would be 40G / 8 / (69+20) = 56.18 Mpps, which splits into 56.18M * 3 = 168.54M flits. Every clock cycle an element reads at most one flit and outputs zero or one flit. This means any FPGA pipeline with clock frequency lower than 168.54 MHz would not be able to achieve line rate. If we waste a cycle between every two packets, the minimum clock frequency would be 224.72 MHz. However, on Stratix V FPGA platform \cite{stratix2012device}, non-trivial hardware logic that accesses registers and local memory can hardly run higher than 200 MHz. Therefore no idle cycles are allowed in elements processing packet content. First, the framework should provide abstractions for programmers to develop fully pipelined network functions. Second, as full compilation of a FPGA program may take hours, the framework should give performance warnings in an early compilation stage if the code cannot be fully pipelined.

\smalltitle{Code reuse.} Many network applications share a common set of elements, for example packet parser, lookup tables and packet modifications. Code of these elements should be reusable and elements should be composable. Software engineers should be able to write many network applications simply by connecting elements in the library.

\smalltitle{Debugging support.} First, as HDL (e.g. Verilog) simulation and debugging is both time consuming and requires extensive hardware knowledge, the framework should be able to compile OpenCL-based ClickNP programs to native x86 code for emulation, and provide traffic generators and receivers to test functionality. Second, as CPU is neither capable of sending or receiving packets at 60 Mpps, we need a FPGA-based network benchmark suite to perform stress testing on the network processor.

\smalltitle{Separation of control plane and data plane.} On one hand, our throughput requirement requires most network packets to be processed through the reconfigurable hardware without any CPU intervention. On the other hand, SDN and NFV applications are usually complicated and have external dependencies. Therefore a clear interface between the control plane and the data plane is mandatory, where data plane programs are written within ClickNP framework and target massive parallelism, and control plane programs need only slight modifications to call our host library and perform on-the-fly reconfigurations.

\smalltitle{Host communication.} Network processors require low-latency and high-throughput interactions with the host machine. In SDN and NFV applications, FPGA needs to send unknown packets to the controller and request a new forwarding rule to be inserted into FPGA. The round-trip time should be as low as possible to reduce end-to-end flow establish time. In packet replay and capture applications, FPGA needs to receive or send Gigabytes of packets from or to the host machine without using the network adapter.

We design ClickNP to meet the above design goals with Catapult FPGA \cite{putnam2014reconfigurable} and Altera OpenCL \cite{singh2011implementing}. In the next section, we will describe the FPGA and OpenCL components, and how we build a toolchain that abstracts away hardware specific details.
}

\egg{
\subsection{FPGA in datacenter}

Conventionally, datacenter operators largely relied on the performance improvements in general-purpose servers
to improve the operation efficiency. This performance improvement rate of servers 
has considerably slowed down recently due to the power limitations~\cite{putnam2014reconfigurable, more-citation}.
This has motivated the adoption of \textit{accelerator} that can be specialized to certain workloads to get efficiency gains.
However, the non-programmable ASIC-based accelerators are undesirable for datacenters due to following two reasons:
Firstly, datacenter operators prefers homogeneous server configurations to minimize the management overhead and also provide
a consistent platform that applications can rely on.
Secondly, services in datacenters evolve extremely rapidly. Waiting for the long release cycle of ASIC chips is undesirable.
%
Therefore, it requires a flexible accelerator that can potentially speed up many applications.
%
GPU and FPGA are two predominate technologies that satisfy this requirement.

% comparison between GPU and FPGA
% power efficiency
Compared with GPU, FPGA is more power efficient. For example, the latest NIVDIA xxx consumes xxx W power, while a high-end
Altera Stratix V consumes xxx W power ( J per op?) \knote{need a citation}. 
% versatile 
Further, FPGA is more versatile. While GPU is mainly designed to achieve \textit{data parallelism} with SPMD (single program, multiple data), 
FPGA can easily achieve both data parallelism and \textit{task parallelism} as different block of LEs can be independently configured to
implement different processing.  
% I/O
Finally, FPGA supports various I/O interface. Normally, GPU can only communicate with PC memory through PCIE bus. 
But FPGA can input or output data from many interfaces like network ports, and is more suitable for processing these 
I/O streams. 

FPGA is a mature technology and becomes inexpensive. A large scale deployment of FPGA in Microsoft datacenter shows 
that a high-end FPGA board increases the total cost of ownership (TCO) of a server by less than 30\%, but can double 
the Bing search efficiency~\cite{putnam2014reconfigurable}.
%
In this paper, we focus on using FPGA to accelerate network functions that are essential to our datacenter networks.
}

\egg{
\smalltitle{inexpensive}

Why I need this as background:
\begin{itemize}
\item Price issue?
\item Power?
\item Transition to FPGA for network functions?
\end{itemize}
}
