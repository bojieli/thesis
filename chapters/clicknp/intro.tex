%!TEX root=main.tex
\section{引言}

本章的主题是网络虚拟化和网络功能加速,在本文中的位置如图 \ref{clicknp:fig:sys-arch} 所示。

\begin{figure}[htbp]
	\centering
	\includegraphics[width=0.8\textwidth]{image/sys_arch.pdf}
	\caption{本章主题:网络虚拟化和网络功能加速,用粗斜线背景的方框标出。}
	\label{clicknp:fig:sys-arch}
\end{figure}

本章作为全文的基础,将提出一个 FPGA 高级语言编程框架,以及主机 CPU 上相应的运行时,并在此基础上实现硬件加速的网络功能,如图 \ref{clicknp:fig:sw-hw-codesign} 所示。

\begin{figure}[htbp]
	\centering
	\includegraphics[width=0.5\textwidth]{image/sw_hw_codesign.pdf}
	\caption{本章在可编程网卡软硬件架构中的位置。}
	\label{clicknp:fig:sw-hw-codesign}
\end{figure}

本章介绍 \name{},一个用于在商用服务器上进行高度灵活和高性能的网络功能处理的FPGA加速平台。
\name{} 通过三个步骤解决 FPGA 的编程挑战。
首先,提供了一个模块化架构,类似于第 \ref{background:sec:network-function} 节介绍的 Click 模型 \cite {kohler2000click},复杂的网络功能可以使用简单的元件组成。
\footnote{这也是系统名称 \textit{Click Network Processor} (ClickNP)的来源。}
其次,\name 元件是用高级C语言编写的,并且是跨平台的。
\name 元件可以通过利用商业高层次综合(High-Level Synthesis,HLS)工具 \cite {vivado,aoc,sdaccel} 在 FPGA 上编译成硬件描述语言和硬件模块,或在 CPU 上使用标准 C++ 编译器编译成机器指令。
最后,高性能PCIE I/O通道可在CPU和FPGA上运行的元件之间提供高吞吐量和低延迟通信。
PCIE I/O通道不仅可以实现CPU-FPGA的联合处理 -- 允许程序员自由地在 CPU 和 FPGA 间划分任务,而且对调试也有很大的帮助,因为程序员可以在主机上轻松地运行有问题的元件,并使用熟悉的软件调试工具。
%虽然以前的工作\cite{Click2NetFPGA}尝试过类似的方法,但与之相比,\name{}实现了两个数量级以上的性能改进。

\name{} 使用一系列优化技术来有效地利用FPGA中的大规模并行性。
首先,\name{} 将每个元件组织成FPGA中的逻辑块,并将它们与先进先出(FIFO)缓冲区连接起来。
因此,所有这些元件块都可以完全并行运行。
对于每个元件,本章仔细编写处理函数以最小化操作之间的依赖关系,从而允许高层次综合工具生成最大并行逻辑。
此外,开发了\textit {延迟写入}和\textit {内存散射}技术来解决读写依赖性和伪内存依赖性,这些问题是现有高层次综合工具无法解决的。
最后,通过在不同阶段仔细平衡操作并匹配其处理速度,可以最大化管道的总体吞吐量。
通过所有这些优化,\name{}实现了每秒高达2亿个数据包的数据包吞吐量 \footnote {\name{} 网络功能的实际吞吐量可能受以太网端口数据速率限制。},并具有超低延迟(对大多数数据包大小,延迟小于 $2 \mu$s)。
与 GPU 和 CPU 上最先进的软件网络功能相比,这大约是10倍和2.5倍的吞吐量增益~\cite {packetshader},同时将延迟分别降低10倍和100倍。

本章实现了\name 工具链,它可以与各种商业高层次综合工具集成 \cite {vivado,aoc},包括Intel FPGA OpenCL SDK 和 Xilinx SDAccel。
本章还实现了大约200个常用元件,其中20\%与Click中的对应元件功能相同,参考Click的代码重新实现。
本章将使用这些元件构建五个演示网络功能:
(1)高速数据包发包和抓包工具,
(2)支持精确匹配和通配符匹配的防火墙,
(3)IPSec网关,
(4)一个可以处理3200万个并发流的四层负载均衡器,
(5)pFabric调度器 \cite {pfabric} 执行严格优先级流调度(flow scheduling),具有40亿个优先级。
评估结果表明,所有这些网络功能都可以通过FPGA大大加速,并且可以在任何数据包大小下使40Gbps的线速饱和,同时具有极低的延迟和可忽略的CPU开销。

总之,本章的贡献是:
(1)\name 语言和工具链的设计与实现;
(2)在FPGA上高效运行的高性能数据包处理模块的设计和实现;
(3)FPGA加速网络功能的设计和评估。
据作者所知,\name 是第一个用于通用网络功能、完全用高级语言编写并能实现40~Gbps线速的FPGA加速数据包处理平台。

\egg{
\smalltitle{Roadmap.} The roadmap of the paper is as follows: \S\ref{clicknp:sec:background} discuss the background.
\S\ref{clicknp:sec:architecture} presents the \name\ architecture and design. 
Our optimizations for FPGA is explained in \S\ref{clicknp:sec:optimization}.
\S\ref{clicknp:sec:impl} presents the implementation details and \name\ 网络功能s are discribed in \S\ref{clicknp:sec:application}.
We evaluate \name\ in \S\ref{clicknp:sec:eval}.
Related work is discussed in \S\ref{clicknp:sec:related} and 
we conclude in \S\ref{clicknp:sec:conclusion} .
}

\egg{
\separate{outline}

The flow: 

Cloud services demand more and more capability. Trend in networking technologies: 40G \arrow 100G~\cite{mellanox-100g}.

Multi-tenancy cloud pushes the network edge/functions to end host~\cite{albert-ons, vmware-multi-tenancy}

Implementing network functions in software (or network function virtualization). Downsides: 1) performance (throughput and latency); and 2) cost (count \# of CPU in use). 

What are the network functions in mind?
\begin{itemize}
\item tunning
\item traffic shaping (rate limiters, load balancer)
\item security (firewall, crypto)
\item management and monitoring
\end{itemize}•


Hardware acceleration is needed. 1) GPU ~\cite{packetshader}; 2) FPGA \cite{netfpga, lockwood2007netfpga} \cite{smartnic}
We also need to mention FPGA is cost efficient~\cite{putnam2014reconfigurable}.

However, the programmability of FPGA is low. High-level Synthesizer could help. ~\cite{bacon2013fpga, feist2012vivado, auerbach2010lime, czajkowski2012opencl, Click2NetFPGA}. 
But these tools are either hard to use for software programmer, or do not have a right interface for network processing.

Why not just use OpenCL?
\begin{itemize}
\item OpenCL is originally designed to allow host program to execute a function (called kernel) on the accelerating devices.
\item With pipe object OpenCL can be used for streaming processing of packet flows, but very cumbersome, requiring manually allocating, distributing and deallocating pipe objects.
\item Code-reuse.  Hard to reuse code.
\end{itemize}•

Programming abstraction: Click model. Familiar, well modeling the packet processing flow.
Key features:
\begin{itemize}
\item elements can be executed in both CPU or FPGA. good for debugging.

\end{itemize}•


Key optimizations in FPGA:
\begin{itemize}
\item memory dependency avoidance. 1) buffer registers; 2) memory stripping; 3) dependency among processing pipelines (?) 
\item Unroll and code expansion. We need maximal iteration loop. (indeterminate, dynamically determined). 
\item Explicit pipelining (control the size of combination logic)
\end{itemize}•


Host library.

CommandHUB (hierarchical cmdhub).

PCIE I/O channels
Batching / Polling / Interrupt / sharing PCIE 

The rest of this paper is organized as follows. Section \ref{clicknp:sec:background} walks through network processor architectures and programming challenges of FPGA, then propose design goals of ClickNP. Section \ref{clicknp:sec:architecture} illustrates the FPGA hardware platform we work on, and provides an overview of the ClickNP toolchain. In addition to programming abstractions for writing elements in OpenCL (section \ref{clicknp:sec:language}), we have also built a library of generic elements (section \ref{clicknp:sec:elements}) including basic connectors, packet parsers, lookup tables, packet modifications and traffic schedulers, which can be linked as a data flow graph using Click-like syntax to perform comprehensive network functions. We evaluate our work via several high performance network applications (section \ref{clicknp:sec:impl_eval}) built with ClickNP framework. Finally we discuss future works (section \ref{clicknp:sec:future}) and conclude (section \ref{clicknp:sec:conclusion}).
}
