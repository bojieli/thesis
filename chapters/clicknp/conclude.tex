%!TEX root=main.tex
\section{本章小结}
\label{clicknp:sec:conclusion}

本文介绍了 \name ,这是一个FPGA加速平台,用于商用服务器中高度灵活和高性能的NF。
\name 使用高级语言完全可编程,并提供网络领域软件程序员熟悉的模块化架构。
\name 支持联合CPU / FPGA数据包处理并具有高性能。
我们的评估表明,与最先进的软件NF相比,\name 将NF的吞吐量提高了10倍,将延迟降低了10倍。
我们的工作提出了一个具体案例,表明FPGA能够加速数据中心的NF。
此外,我们证明了FPGA的高级编程实际上是可行和实用的。
然而,FPGA编程的一个限制是编译时间相当长,例如1小时约2小时,主要是由于HDL综合工具。
\name 通过其跨平台能力缓解这种痛苦,并希望通过在CPU上运行元素来检测大多数错误。
但是,从长远来看,HDL综合工具应该进行优化,以大大缩短编译时间。





\begin{sidewaystable}[h]
	\centering
	\caption{A selected set of key elements in \name. }
	\label{clicknp:tab:elements}
	\scalebox{0.9}{
		\begin{tabular}{l|l|l|l|l|l|l|r|r}
			\toprule
			&	&	& \multicolumn{4}{c|}{Performance} & \multicolumn{2}{c}{Resource (\%)} 			\\
			\cline{4-7} \cline{8-9}  
			
			元件 	& 配置 & 优化 & 最高频率 (MHz) & 峰值吞吐量 & 加速比 (FPGA/CPU) & 延迟(时钟周期) & LE & BRAM \\
			\midrule
			L4\_Parser (A1-5)  & N/A & REG & 221.93 & 113.6 Gbps & 31.2x / 41.8x & 11 & 0.8\% & 0.2\% \\
			IPChecksum (A1-4) & N/A & UL & 226.8 & 116.1 Gbps & 33.1x / 55.1x & 18 & 2.3\% & 1.3\% \\
			NVGRE\_Encap (A1,4) & N/A & REG, UL & 221.8 & 113.6 Gbps & 35.5x / 42.9x & 9 & 1.5\% & 0.6\% \\
			\midrule
			AES\_CTR (A3) & 16B block & UL & 217.0 & 27.8 Gbps & 79.9x / 255x & 70 & 4.0\% & 23.1\% \\
			SHA1 (A3) & 64B block & MS, UL & 220.8 & 113.0 Gbps & 157.5x / 83.1x & 105 & 7.9\% & 6.6\% \\
			\midrule
			\midrule
			CuckooHash (A2) & 128K entries & MS, UL, DW & 209.7 & 209.7 Mpps & 43.6x / 57.5x & 38 & 2.0\% & 65.5\% \\
			HashTCAM (A2) & 16 x 1K & MS, UL, DW & 207.4 & 207.4 Mpps & 155.9x / 696x & 48 & 18.7\% & 22.0\% \\
			LPM\_Tree (A2) & 16K entries & MS, UL, DW & 221.8 & 221.8 Mpps & 34.5x / 45.2x & 181 & 4.3\% & 13.2\% \\
			FlowCache (A4) & 4-way, 16K & MS, DW & 105.6 & 105.6 Mpps & 55.8x / 21.5x & 27 & 5.6\% & 46.9\% \\
			\midrule
			
			SRPrioQueue (A5) & 32 Pkts buffer & REG, UL & 214.5 & 214.5 Mpps & 150.3x / 28.6x & 41 & 2.6\% & 0.6\% \\
			RateLimiter (A1,5) & 16K flows & MS, DW & 141.5 & 141.5 Mpps & 7.0x / 65.3x & 14 & 16.9\% & 14.1\% \\
			\bottomrule
			
			\multicolumn{9}{l} {\textbf{Optimization method.} REG=Using Registers; MS=Memory Scattering; UL=Unroll Loop; DW=Delay Write.} \\
			\multicolumn{9}{l} {\parbox{\textwidth}{The \textbf{Speedup} column compares the performance between
					the optimized version and our earlier implementation without applying techniques discussed in \S\ref{clicknp:sec:optimization} as well as a CPU implementation.}}
			
		\end{tabular} 
		
	}
\end{sidewaystable}
