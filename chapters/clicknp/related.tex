%!TEX root=main.tex

\section{Related Work}
\label{clicknp:sec:related}

% software network functions
Software NFs have great flexibility and scalability.
% Software switches 
Early studies mainly focus on software-based packet forwarding~\cite{routebricks, Egi:2008:THP:1544012.1544032 }. 
They show that multi-core x86 CPU can forward packets at near 10Gbps per server
and the capacity can scale by clustering more servers. 
%
Recently, many  systems have been designed to implement various types of NFs~\cite{comb, Greenhalgh:2009:FPR:1517480.1517484, martins2014clickos}. 
Similarly, all of these systems exploit the multi-core parallelism in CPUs to achieve close to 10Gbps 
throughput per machine, and scale out to use more machines when higher capacity is needed. 
%
Ananta~\cite{ananta} is a software load-balancer deployed in Microsoft datacenters to provide 
cloud-scale load-balancing service.
%
While software NFs can scale out to provide more capacity, doing so adds 
considerable costs in both CAPEX and OPEX~\cite{ananta, duet}.

% accelerations
To accelerate software packet processing, previous work has proposed using GPU~\cite{packetshader},
specialized network processor (NP)~\cite{cavium,netronome }, and hardware switches~\cite{duet}.
% GPU
GPU is primarily designed for graphic processing and recently extended to other applications with massive 
data parallelism. GPU is more suitable for batch operations. Han, \etal~\cite{packetshader}, show that using GPU
can achieve 40Gbps packet switching speed. 
However, batch operations incur high delay. 
For example, the forwarding latency reported in \cite{packetshader} is about $200\mu s$, two orders of 
magnitude larger than \name.
Compared to GPU, FPGA is more flexible and can be reconfigured to capture data and pipeline parallelisms, both of which are very common in NFs.
% NP
NPs, however, are specialized to handle network traffic and have many hard-wired network accelerators.
In contrast, FPGA is a general computing platform. 
Beside NFs, FPGA have many other applications in datacenters, making it more attractive 
to deploy at scale~\cite{putnam2014reconfigurable}.
%
Hardware switch has limited functionality and its applications are very restricted~\cite{duet}.

FPGA is a mature technology and recently has been deployed to accelerate datacenter services, including NFs~\cite{putnam2014reconfigurable, smartnic, rubow2010chimpp, lavasani2012compiling}.
It is well recognized that the programmability of FPGA is low and there is a 
rich body of literature on improving it, 
by providing high-level programming abstractions~\cite{bluespec, auerbach2010lime, bacon2013fpga, singh2011implementing, bachrach2012chisel, wester2015transformation}.
%
Gorilla \cite{lavasani2012compiling} proposes a domain-specific high-level language for packet switching on FPGA.
Chimpp \cite{rubow2010chimpp}, however, tries to introduce Click model into HDL 
to develop modular router.
\name\ works along this direction and is complimentary to previous work.
\name\ targets NFs in datacenters, and addresses the programmability issue by
providing a highly flexible, modularized architecture and leveraging commercial HLS tools.

% add comments on rubow2010chimpp,
The work most related to ours is the Click2FPGA toolchain~\cite{Click2NetFPGA}, which compiles an entire Click configuration to FPGA. 
Its performance, however, is much lower than \name\ and it lacks support for 
joint CPU/FPGA packet processing.
\egg{
High performance PCIe communication designs between CPU and co-processors are explored in~\cite{lee2015flosis, nam2015haetae}. 
ClickNP follows this line of work and proposes a design for efficient joint CPU-FPGA processing.}
%
To the best of our knowledge, \name\ is the first FPGA-accelerated platform for general NFs
processing at 40Gbps line rate, and completely written in high-level language.
